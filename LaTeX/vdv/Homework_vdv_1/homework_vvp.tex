\documentclass[a4paper]{article}
\usepackage[T2A]{fontenc}
\usepackage[russian]{babel}
\usepackage{amsmath}
\usepackage{amssymb}
\usepackage{graphicx}
\usepackage{float}
\usepackage{hyperref}
\usepackage{physics}



\begin{document}

\section{First problem}

Для перевода в полярную систему координат используем формулы:\\
\begin{centering}
	$(1)\mathbf{r}={r}\cos\theta\cdot{\mathbf{e_x}}+{r}\sin\theta\cdot{\mathbf{e_y}}$\\
\end{centering}
\begin{itemize}
	\item $P_1: x^2+y^2=r^2 \Rightarrow r=\sqrt{x^2+y^2}=\sqrt{2}$; $\theta=\arccos\dfrac{x}{r}\Rightarrow \theta=\dfrac{\pi}{4}$
	\item $P_2: x^2+y^2=r^2 \Rightarrow r=\sqrt{9+16}=5$; $\theta=\arccos\dfrac{x}{r}\Rightarrow \theta=\arccos\dfrac{-3}{5}$
	\item $P_3: x^2+y^2=r^2 \Rightarrow r=\sqrt{10685}$; $\theta=\arccos\dfrac{-101}{\sqrt{10685}}$
	\item $P_4: x^2+y^2=r^2 \Rightarrow r=\sqrt{13}$; $\theta=\arccos\dfrac{-2}{\sqrt{13}}$
\end{itemize}
\section{Second problem}
\begin{itemize}
	\item Для декартовых координат: $x^2+y^2=R^2$
	\item Для полярных координат: $r=R$, это можно получить подстановкой в (1)
\end{itemize}

\section{Third problem}
\begin{itemize}
	\item Длину вектора можно посчитать через Th Пифагора: ${r}=\sqrt{{r_x}^2+{r_y}^2} \Rightarrow a=\sqrt{10}, b=\sqrt{10}$
	\item Угол между векторами можно посчитать черерз скалярное произведение векторов: $\theta= \arccos\dfrac{\mathbf{a}\cdot\mathbf{b}}{|{a}|\cdot|{b}|}\Rightarrow \theta=\arccos\dfrac{3*1+(-1*3)}{10}=\pi/2$
\end{itemize}
\section{Fourth problem}
$x=a\cos t, y=a\sin t$  $\,$ хммм, это же окружность с центром в (0,0) и радиусом $a$, так как $x^2+y^2=a^2, \,\mathbf{e_r}$ сонаправлен с $\mathbf{a}$ и перпендикулярен $\mathbf{e_\varphi} \Rightarrow \mathbf{a}={a}\cdot\mathbf{e_r}$
\end{document}

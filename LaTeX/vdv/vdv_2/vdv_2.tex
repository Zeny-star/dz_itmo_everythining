\documentclass{report}

\usepackage[T2A]{fontenc}
\usepackage[russian]{babel}
\usepackage{graphicx}
\usepackage{float}
\usepackage{hyperref}
\usepackage{amsmath}
\usepackage{diffcoeff,amssymb}
\usepackage{mathtools}
\usepackage[normalem]{ulem}

\input{preamble}
\input{macros}
\input{letterfonts}



\title{\Huge{Вдв}\\ Решение дз №2}
\author{\huge{Евгений Турчанин}}
\date{}
\begin{document}
\maketitle

\qs{}{
Найдите вещественные и мнимые части следующих комплексных чисел, а также их комплексно сопряженные
$$
\text { 1) } \dfrac{1}{1+i} \text {; 2) }\left(\dfrac{1+i}{1-i}\right)^{3} \text {; 3) }\left(\dfrac{1}{2}+i \dfrac{\sqrt{3}}{2}\right)^{3} \text {; 4) }\left(\dfrac{i^{9}+2}{i^{23}+1}\right)^{2} \text { 5) } 10 \exp \left\{\dfrac{3 \pi}{4} i\right\}
$$
}

\sol 
\begin{enumerate}
	\item $\dfrac{1}{1+i} =\dfrac{1-i}{2} \Rightarrow Re=\dfrac{1}{2},\, Im=-\dfrac{1}{2}, z^*=\dfrac{1}{2}+i/2$
\item $\left(\dfrac{1+i}{1-i}\right)^3 =\dfrac{(1+i)^2(1+i)^2(1+i)^2}{(1-i)^3(1+i)^3} =\dfrac{-8i}{8}=-i \Rightarrow Re=0,\, Im=-1, \, z^*=i$
\item $\left(\dfrac{1}{2}+i \dfrac{\sqrt{3}}{2}\right)^{3}=\dfrac{1}{2}^3 +3\cdot\dfrac{1}{4}\cdot\dfrac{\sqrt{3}}{2}i-3\cdot\dfrac{1}{2}\cdot\dfrac{3}{4}-\left(\dfrac{\sqrt{3}}{2}\right)^3i \Rightarrow Re=-1,\, Im=0,\,z^*=-1$
\item $\left(\dfrac{i^{9}+2}{i^{23}+1}\right)^{2}=\left(\dfrac{2+i}{1-i}\right)^2=\dfrac{4i+3}{-2i}=-2+\dfrac{3}{2}i \Rightarrow Re=-2,\, Im=\dfrac{3}{2}, \, z^*=-2 -\dfrac{3}{2}i$
\item $10\exp \left\{\dfrac{3 \pi}{4} i\right\} \Rightarrow r=10 \varphi=\dfrac{3\pi}{4} \Rightarrow 10\exp \left\{\dfrac{3 \pi}{4} i\right\}=10(\cos{3\pi/4+i\sin{3\pi/4}}) \Rightarrow Re=-5\sqrt{2},\, Im=5\sqrt{2}, \, z^*=-5\sqrt2-5\sqrt2 i$
\end{enumerate}

\clm{}{}{\bf{Ответ:}
\begin{enumerate}
\item $Re=\dfrac{1}{2}, Im=-\dfrac{1}{2}, z^*=\dfrac{1}{2}-i/2$
\item $Re=0, Im=-1, z^*=i$
\item $Re=-1, Im=0, z^*=-1$
\item $Re=-2, Im=\dfrac{3}{2}, z^*=-2 -\dfrac{3}{2}i$
\item $Re=-5\sqrt{2}, Im=5\sqrt{2}, z^*=-5\sqrt2-5\sqrt2 i$
\end{enumerate}

}


\qs{}{
Перепишите следующее комплексные числа через их модуль и фазу (аргумент)
$$
\text { 1) } 1+i^{321} \text {; 2) } \dfrac{1-i}{1+i} \text {; 3) }(1+i)^{8}(1-i \sqrt{3})^{-6} \text {; 4) } 1+\cos \dfrac{\pi}{7}+i \sin \dfrac{\pi}{7}
$$
}

\sol
\begin{enumerate}
\item $1+i^{321} = 1+i \Rightarrow r=\sqrt{2};\, \varphi=\dfrac{\pi}{4}$
\item $\dfrac{1-i}{1+i} =\dfrac{(1-i)^2}{2}=\dfrac{-2i}{2}=-i \Rightarrow r=1;\, \varphi=-\dfrac{\pi}{2}$
\item $(1+i)^{8}(1-i \sqrt{3})^{-6}=\left(\dfrac{(1+i)^4}{(1-i\sqrt{3})^{-3}}\right)^2=\left(\dfrac{-4}{1-3\sqrt{3}i-9+3^{3/2}i}\right)^2=\dfrac{1}{4} \Rightarrow r=\dfrac{1}{4}; \, \varphi=0$
\item $1+\cos \dfrac{\pi}{7}+i \sin \dfrac{\pi}{7} \Rightarrow a=1+\cos{\dfrac{\pi}{7}}; \, b=\sin{\dfrac{\pi}{7}} \Rightarrow r=\sqrt{1+2\cos{\dfrac{\pi}{7}}+\cos^2{\dfrac{\pi}{7}}+\sin^2{\dfrac{\pi}{7}}}=
	\sqrt{2+2\cos{\dfrac{\pi}{7}}} \Rightarrow \varphi=\arctan{\dfrac{\sin{\pi/7}}{1+\cos{\pi/7}}}$
\end{enumerate}

\clm{}{}{\bf{Ответ:}
\begin{enumerate}
\item $r=\sqrt{2};\, \varphi=\dfrac{\pi}{4}$
\item $r=1;\, \varphi=-\dfrac{\pi}{2}$
\item $r=\dfrac{1}{4};\, \varphi=0$
\item $r=\sqrt{2+2\cos{\dfrac{\pi}{7}}} ;\, \varphi=\arctan{\dfrac{\sin{\pi/7}}{1+\cos{\pi/7}}}$
\end{enumerate}}

\qs{}{
Найти все решения следующих уравнений:
$$
\begin{aligned}
& \text { 1) } z^{2}=3-4 i \\
& \text { 2) } z^{8}=1+i \\
& \text { 3) } z^{6}=64 \\
& \text { 4) } z^{7}+1=0
\end{aligned}
$$
}

\sol

\begin{enumerate}
	\item $z^2=3-4i \Rightarrow a=3;\; b=-4 \Rightarrow r=5 \Rightarrow \cos{\varphi}=\dfrac{3}{5};\; \sin{\varphi}=-\dfrac{4}{5} \Rightarrow z=\sqrt{5}(\cos{\dfrac{\varphi+2\pi k}{2}}+i\sin{\dfrac{2\varphi+2\pi k}{2}}), $\\
		Рассмотрим два случая:
		\begin{itemize}
			\item $k=0$: $z=\sqrt{5}(\cos{\varphi/2}+i\sin{\varphi/2})=\sqrt{5}\left(\dfrac{2}{\sqrt{5}}+i\dfrac{1}{\sqrt{5}}\right) \Rightarrow z=2+i$
			\item $k=1$: $z=\sqrt{5}(\cos{(\varphi/2+\pi)}+i\sin{(\varphi/2+\pi)}) \Rightarrow z=-2+i$
		\end{itemize}
		Очевидно что, других корней нету, тк $\cos$ и $\sin$ --- переодические фунции.
\item $z^8=1+i \Rightarrow a=1;\; b=i \Rightarrow r=\sqrt{2};\; \varphi=\dfrac{\pi}{4}+2\pi k \Rightarrow$
	$z=2^{1/16}\exp{\left[i(\pi/4+2\pi k)/8\right]}$\\
	Рассмотрим \sout{два случая} несколько...:
	\begin{itemize}
		\item $k=0$: $z=2^{1/16}\exp{\left[i(\pi/4)/8\right]} \Rightarrow z=2^{1/16}\exp{\left[i\pi/32\right]}$
		\item $k=1$: $z=2^{1/16}\exp{\left[i(\pi/4+2\pi)/8\right]} \Rightarrow z=2^{1/16}\exp{\left[9i\pi/32\right]}$
		\item $k=2$: $z=2^{1/16}\exp{\left[i(\pi/4+4\pi)/8\right]} \Rightarrow z=2^{1/16}\exp{\left[17i\pi/32\right]}$
		\item $k=3$: $z=2^{1/16}\exp{\left[i(\pi/4+6\pi)/8\right]} \Rightarrow z=2^{1/16}\exp{\left[25i\pi/32\right]}$
		\item $k=4$: $z=2^{1/16}\exp{\left[i(\pi/4+8\pi)/8\right]} \Rightarrow z=2^{1/16}\exp{\left[33i\pi/32\right]}$
		\item $k=5$: $z=2^{1/16}\exp{\left[i(\pi/4+10\pi)/8\right]} \Rightarrow z=2^{1/16}\exp{\left[41i\pi/32\right]}$
		\item $k=6$: $z=2^{1/16}\exp{\left[i(\pi/4+12\pi)/8\right]} \Rightarrow z=2^{1/16}\exp{\left[49i\pi/32\right]}$
		\item $k=7$: $z=2^{1/16}\exp{\left[i(\pi/4+14\pi)/8\right]} \Rightarrow z=2^{1/16}\exp{\left[57i\pi/32\right]}$
	\end{itemize}

\item $z^6=64 \Rightarrow a=64;\; b=0 \Rightarrow r=64 \Rightarrow \cos{\varphi}=1;\; \sin{\varphi}=0 \Rightarrow \varphi=0 \Rightarrow z=64^{1/6}
	\exp{\left[i(0+2\pi k)/6\right]}$\\
	Рассмотрим несколько случаев:
	\begin{itemize}
		\item $k=0$: $z=64^{1/6}\exp{\left[i(0)/6\right]} \Rightarrow z=2$
		\item $k=1$: $z=64^{1/6}\exp{\left[i(2\pi)/6\right]} \Rightarrow z=2\exp{\left[i\pi/3\right]}$
		\item $k=2$: $z=64^{1/6}\exp{\left[i(4\pi)/6\right]} \Rightarrow z=2\exp{\left[2i\pi/3\right]}$
		\item $k=3$: $z=64^{1/6}\exp{\left[i(6\pi)/6\right]} \Rightarrow z=2\exp{\left[i\pi\right]}$
		\item $k=4$: $z=64^{1/6}\exp{\left[i(8\pi)/6\right]} \Rightarrow z=2\exp{\left[4i\pi/3\right]}$
		\item $k=5$: $z=64^{1/6}\exp{\left[i(10\pi)/6\right]} \Rightarrow z=2\exp{\left[5i\pi/3\right]}$
	\end{itemize}
\item $z^7+1=0 \Rightarrow a=-1;\; b=0 \Rightarrow r=1 \Rightarrow \cos{\varphi}=-1;\; \sin{\varphi}=0 \Rightarrow \varphi=\pi \Rightarrow z=\exp{\left[i(\pi+2\pi k)/7\right]}$\\
	Рассмотрим несколько случаев:
	\begin{itemize}
		\item $k=0$: $z=\exp{\left[i(\pi)/7\right]} \Rightarrow z=\exp{\left[i\pi/7\right]}$
		\item $k=1$: $z=\exp{\left[i(\pi+2\pi)/7\right]} \Rightarrow z=\exp{\left[3i\pi/7\right]}$
		\item $k=2$: $z=\exp{\left[i(\pi+4\pi)/7\right]} \Rightarrow z=\exp{\left[5i\pi/7\right]}$
		\item $k=3$: $z=\exp{\left[i(\pi+6\pi)/7\right]} \Rightarrow z=\exp{\left[7i\pi/7\right]}$
		\item $k=4$: $z=\exp{\left[i(\pi+8\pi)/7\right]} \Rightarrow z=\exp{\left[9i\pi/7\right]}$
		\item $k=5$: $z=\exp{\left[i(\pi+10\pi)/7\right]} \Rightarrow z=\exp{\left[11i\pi/7\right]}$
		\item $k=6$: $z=\exp{\left[i(\pi+12\pi)/7\right]} \Rightarrow z=\exp{\left[13i\pi/7\right]}$
	\end{itemize}
\end{enumerate}
\clm{}{}{\bf{Ответ:}\\
см выше
}


\qs{}{
Даны три комплексные числа \( z_1 = x_1 + iy_1 \), \( z_2 = x_2 + iy_2 \), \( z_3 = x_3 + iy_3 \). 
Необходимо найти вещественную и мнимую часть следующего выражения в терминах \( x_1, x_2, x_3 \), \( y_1, y_2, y_3 \):
\[
	z_1 z_2^* + z_2 z_3^* + z_3 z_1^* +\dfrac{1}{z_1^* - z_2 + z_3^*}
\]
Где \( z^* \) обозначает комплексно-сопряженное число.
}



Найдем вещественную и мнимую часть каждого выражения:

\begin{enumerate}
	\item $z_1z_2^*=\underbrace{x_1x_2+y_1y_2}_{\text{Re}}+\underbrace{i(x_2y_1-x_1y_2)}_{\text{Im}}$
	\item $z_2z_3^*=\underbrace{x_2x_3+y_2y_3}_{\text{Re}}+\underbrace{i(x_3y_2-x_2y_3)}_{\text{Im}}$
	\item $z_3z_1^*=\underbrace{x_3x_1+y_3y_1}_{\text{Re}}+\underbrace{i(x_1y_3-x_3y_1)}_{\text{Im}}$
	\item $\dfrac{1}{z_1^* - z_2 + z_3^*}=\dfrac{1}{x_1-x_2+x_3+i(-y_1-y_2-y_3)}=\underbrace{\dfrac{x_1-x_2+x_3}{(x_1-x_2+x_3)^2+(y_1+y_2+y_3)^2}}_{\text{Re}} +i\underbrace{\dfrac{(y_1+y_2+y_3)}{(x_1-x_2+x_3)^2+(y_1+y_2+y_3)^2}}_{\text{Im}}$
\end{enumerate}
Объеденяя эти выражения получим:
\begin{center}
\begin{equation}
Re=x_1x_2+y_1y_2+x_2x_3+y_2y_3+x_3x_1+y_3y_1+\dfrac{x_1-x_2+x_3}{(x_1-x_2+x_3)^2+(y_1+y_2+y_3)^2}
\end{equation}
\begin{equation}
Im=x_2y_1-x_1y_2+x_3y_2-x_2y_3+x_1y_3-x_3y_1 +\dfrac{(y_1+y_2+y_3)}{(x_1-x_2+x_3)^2+(y_1+y_2+y_3)^2}
\end{equation}

\clm{}{}{\bf{Ответ:}\\
$Re=x_1x_2+y_1y_2+x_2x_3+y_2y_3+x_3x_1+y_3y_1+\dfrac{x_1-x_2+x_3}{(x_1-x_2+x_3)^2+(y_1+y_2+y_3)^2}
$

$Im=x_2y_1-x_1y_2+x_3y_2-x_2y_3+x_1y_3-x_3y_1 +\dfrac{(y_1+y_2+y_3)}{(x_1-x_2+x_3)^2+(y_1+y_2+y_3)^2}
$

}




\end{center}
\end{document}

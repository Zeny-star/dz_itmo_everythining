\documentclass{report}

\usepackage[T2A]{fontenc}
\usepackage[russian]{babel}
\usepackage{amsmath}
\usepackage{amssymb}
\usepackage{graphicx}
\usepackage{float}
\usepackage{hyperref}

\input{preamble}
\input{macros}
\input{letterfonts}

\title{\Huge{Матан}\\ Задача про Волка}
\author{\huge{Евгений Турчанин}}
\date{}

\begin{document}

\maketitle
\dfn{}{Точка A движется равномерно со скоростью $\mathbf{v}$ так, что вектор $\mathbf{v}$ все время "нацелен" на точку В, которая движется прямолинейно и равномерно со скоростью $u<v$. В начальный момент $\mathbf{v} \perp \mathbf{u}$ и расстояние между точками равно $l$. Через сколько времени точки встретятся и какая траектория движения у точки A?}
\sol Рассмотрим малое перемещение $\delta{x}$
\end{document}

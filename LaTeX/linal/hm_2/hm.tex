\documentclass{report}

\usepackage[T2A]{fontenc}
\usepackage[russian]{babel}
\usepackage{graphicx}
\usepackage{float}
\usepackage{hyperref}
\usepackage{amsmath}
\usepackage{diffcoeff,amssymb}
\usepackage{mathtools}
\usepackage[normalem]{ulem}


\input{preamble}
\input{macros}
\input{letterfonts}



\title{\Huge{Линейная алгебра}\\ \Huge{Дз №2}}

\author{\huge{Евгений Турчанин}}
\date{}
\begin{document}
\maketitle

\qs{}{
\textbf{1.4 Комплексные числа}

\begin{enumerate}
	\item Найдите значение $(7-10i-(-3-9i))(-3+18i-(-9+9i))-(-3-9i)^2$
	\item Переверите число из экспоненциальной формы в алгебраическую с точностьо не менее двух знаков после запятой: $8e^{(0.55\pi)}$
	\item Перевидите число $-3+10і$ из алгебраической формы в экспонинциальную, ответ округлите до двух знаков после запятой
	\item Найдите эначение $\sqrt[3]{2}(\cos(6.012)+i\sin(6.012))$ в тригонометрической форме с наименьшим значением угла. Ответ нужно уазать с точностью не менее двух знаков после запятой
	\item Найти комплексное число $z^{-1}$, обратное к данному $z=-3-9i$ по умножению с точностью не менее двух знаков после запятой
\end{enumerate}



\end{document}

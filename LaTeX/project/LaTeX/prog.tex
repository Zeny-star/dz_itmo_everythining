\documentclass[hoptionsi, twocolumn]{revtex4-2}
\usepackage[english,russian]{babel}
\usepackage[utf8]{inputenc}
\usepackage{natbib}
\usepackage{setspace}
\usepackage{graphicx}
\usepackage{dcolumn}
\usepackage{bm}
\usepackage{geometry}
\usepackage{graphicx}
\usepackage{hyperref}
\usepackage{amsmath}
\usepackage{float}
\usepackage{wrapfig}

\hypersetup{
  colorlinks   = true, %Colours links instead of ugly boxes
  urlcolor     = blue, %Colour for external hyperlinks
  linkcolor    = blue, %Colour of internal links
  citecolor   = red %Colour of citations
}

\begin{document}

\section{\textbf{Численное моделирование}}

Для численного моделировая использовался язык python, с пакетами scipy, numpy, matplotlib. Для численного решения дифференциальных уравнений использовался метод \href{https://docs.scipy.org/doc/scipy/reference/generated/scipy.integrate.solve_ivp.html}{solveivp}, с методом интегрирования \href{https://docs.scipy.org/doc/scipy/reference/generated/scipy.integrate.solve_ivp.html#r179348322575-1}{DOP853}, данный метод подходит, так как величина шага в нем $10^{-7}$, что на порядок меньше чем разность аппроксимации.
\bibliography{plus} 
\end{document}

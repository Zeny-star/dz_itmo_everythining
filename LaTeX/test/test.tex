\documentclass[hoptionsi, twocolumn]{revtex4-2}
\usepackage[english,russian]{babel}
\usepackage{setspace}
\usepackage{graphicx}
\usepackage{dcolumn}
\usepackage{bm}
\usepackage[utf8]{inputenc}

\title{Изучение движения вертикально падающих тел}
\begin{document}
\author{Турчанин Е., Солдатова В., Горбушкин Г. и Мнацаканов Ф.}

\maketitle

\section{Введение}
В начале XVII века Иоагном Кеплером на основе длительных астрономических наблюдений Тихо Браге были открыты три эмпирических соотношения. Согласно первому закону, все тела солнечной системы движутся по эллипсу. Движение вертикально падающего тела в гравитационном поле также представляет собой эллипс. Соответственно, по точно такой же траектории будет двигаться вертикально падающее тело (т.е. тело с нулевой начальной скоростью в верхней точке своей траектории).
Кроме того, если движение происходит в неинерциальной системе отсчета, смещение обусловлено не только эллиптичностью траектории, но и воздействием инерциальных сил (например, силой Кориолиса). Изучение такого движения на Земле представляет особый интерес, так как демонстрирует неинерциальность системы отсчета, связанной с Землей. Кроме того, изучение этого отклонения имеет практическую значимость: можно определить параметры траектории падающего тела, при которых уже нельзя пренебрегать этими отклонениями.

Исследование траекторий свободного падения тела в гравитационном поле Земли проводилось учеными, такими как J.M. Potgieter, Edward A. Desloge, Elie Belorizky и другими. В их статьях рассматривалось вертикальное падение тела в гравитационном поле Земли с нулевой начальной скоростью (с определением соответствующих смещений по долготе и широте).

В рамках представленного проекта будет реализована аналогичная задача. Будет изучено падение тел как с нулевой, так и ненулевой начальными скоростями; рассмотрено влияние значения ускорения свободного падения на отклонение тела от вертикали; определено при каких параметрах (высоты, начальной скорости) этими смещениями можно пренебречь. Так как реализация эксперимента имеет ряд сложностей (к примеру, необходимость минимизировать сопротивление воздуха), поэтому данные будут получены путем моделирования для наглядной демонстрации.


\end{document}

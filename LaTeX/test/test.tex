\documentclass{report}

\usepackage[T2A]{fontenc}
\usepackage[russian]{babel}
\usepackage{graphicx}
\usepackage{float}
\usepackage{hyperref}
\usepackage{amsmath}
\usepackage{diffcoeff,amssymb}
\usepackage{mathtools}
\usepackage[normalem]{ulem}

\input{preamble}
\input{macros}
\input{letterfonts}

\setcounter{secnumdepth}{0}
\title{\Huge{Матан Лаба}}
\author{\huge{Павел Андреев, Григорий Горбушкин, Евгений Турчанин}}
\date{}
\begin{document}
\maketitle

\qs{}{
Объяснить, почему разность назад $f_-(x_0)$ – разумное приближение производной в точке $x_0$.
}
По определению
\begin{equation}
f'(x_0)=\lim_{h\rightarrow{0}}{\dfrac{f(x_0+h)-f(x_0)}{h}},
\end{equation}
\begin{equation}
f'(x_0-h)=\lim_{h\rightarrow{0}}{\dfrac{f((x_0-h)+h)-f(x_0-h)}{h}},
\end{equation}
При $h\rightarrow{0}$, $f'(x_0-h)\rightarrow{f'(x_0)}$. Значит, $\lim_{h\rightarrow{0}}{\dfrac{f((x_0-h)+h)-f(x_0-h)}{h}}=\lim_{h\rightarrow{0}}{\dfrac{f(x_0+h)-f(x_0)}{h}}=f'(x_0)$
\label{eq:1}

\begin{equation}
  f(x_0-h)=f(x_0)-hf'(x_0)+\dfrac{h^2}{2}f''(\xi)\text{,}
  \label{eq:2}
\end{equation}
где $\xi \in [x_0 -h, x_0]$, тогда, подставляя \ref{eq:2} в \ref{eq:1}
\begin{equation}
  \left|f'(x_0)-f_-(x_0)\right|\leq \dfrac{M_2h}{2} 
\end{equation}
\qs{}{
Формально показать, как получаются формулы для оценки погрешности в случае приближения производной первой (односторонней) разностью.
}


\qs{}{
Объяснить, почему центральная разность $f_{\cdot 1}(x_0)$ – разумное приближение производной в точке $x_0$.
}

\qs{}{
Формально показать, как получаются формулы для оценки погрешности в случае приближения производной центральной разностью.
}

\qs{}{
Узнать, как хранятся числа, скажем, в python. Узнать, что такое машинная точность. Объяснить, почему в python 0.1+0.2 != 0.3.
}

\qs{}{
\begin{enumerate}
  \item Выбрать любую дважды дифференцируемую функцию $f$.
  \item Аппроксимировать производную с помощью разностей $f_{\pm}(x_0)$, $f_{\circ 1}(x_0)$.
  \item Построить график зависимости $\Delta(x, \varepsilon)$ от $h$, $h \in (10^{-20},\ 1)$, \[\Delta(x, h)=\left| f'(x)-f_{\pm, \circ_1}\right|\text{,} \] при разных $x$ (шкала по оси $y$ – логарифмическая!).
\end{enumerate}
}
\end{document}

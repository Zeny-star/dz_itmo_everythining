\documentclass{report}

\usepackage[T2A]{fontenc}
\usepackage[russian]{babel}
\usepackage{amsmath}
\usepackage{amssymb}
\usepackage{graphicx}
\usepackage{float}
\usepackage{hyperref}
\documentclass{report}

\usepackage{diffcoeff}


\input{preamble}
\input{macros}
\input{letterfonts}




\title{\Huge{Матан}\\Решение 2-ой Домашки}
\author{\huge{Евгений Турчанин}}
\date{}

\begin{document}
\dfn{2.38}{Тело бросают вертикально вверх в вязкой среде. Сила вязкого трения пропорциональна скорости движения тела. Вычислить время $t_1$ подъема(спуска) тела на максимальную высоту его полета вверх и сравнить его со временем $t_0$ подъема в отсутствие трения. Начальная скорость тела в обоих случаях одинакова.}

\sol Запишем уравнение движения: \begin{center}$-ma=mg+\alpha v \Rightarrow -\diff{v}{t}=g+ \frac{\alpha v}{m} \Rightarrow $\\ \vspace{3ex} $\Rightarrow -\int \dfrac{\dl v}{g+\frac{\alpha v}{m}}=\int \dl t$ \end{center}

Поделим, и домножим на $\frac{\alpha}{m}$, и занесем g под $\dl v:$ \begin{center}$ -\frac{m}{\alpha} \int \dfrac{\dl{ (\frac{\alpha v}{m}+g})}{g+\frac{\alpha v}{m}}=\int \dl t \Rightarrow t_1=\frac{m}{\alpha}
\ln \left(1+\dfrac{\alpha v_0}{mg}\right)$ - тк $v$ конечное = 0 \end{center}

Найдем высоту подъема H:\begin{center} $-\dfrac{\dl v}{\dl t}=g+ \dfrac{\alpha v}{m} \Rightarrow -\int_{v_0}^{0} \dl v =\int_{0}^{t_1} g\dl t+\int_{0}^{H} \dfrac{\alpha}{m} \dl x \Rightarrow$ \end{center}

\begin{center}
    $\Rightarrow H=\dfrac{(v_0+gt_1)m}{\alpha}=\dfrac{v_0}{\alpha}+
\dfrac{gm}{\alpha^2} \ln (1+\dfrac{\alpha v_0}{mg})$
\end{center}

\clm{}{}{\bf{Ответ:} \\ $t_1=\frac{m}{\alpha}\ln \left(1+\dfrac{\alpha v_0}{mg}\right)$; $H=\dfrac{v_0}{\alpha}+\dfrac{gm}{\alpha^2} \ln (1+\dfrac{\alpha v_0}{mg})$}
\end{document}

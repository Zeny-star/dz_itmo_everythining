\documentclass{article}
\usepackage{amsmath}
\usepackage{graphicx}
\usepackage[T2A]{fontenc}
\usepackage[utf8]{inputenc}
\usepackage[russian,english]{babel}

\title{Рабочий протокол и отчет по лабораторной работе №1.03: \\ Изучение центрального соударения двух тел}
\author{Студент: \\ Преподаватель: }
\date{}

\begin{document}

\maketitle

\section*{Цель работы}
Исследование упругого и неупругого центрального соударения тел на примере тележек, движущихся с малым трением. 
Исследование зависимости ускорения тележки от приложенной силы и массы тележки.

\section*{Задачи}
\begin{itemize}
    \item Расчет импульсов тел в каждом из опытов, относительных изменений импульса и энергии.
    \item Вычисление силы натяжения нити при проведении опытов с разной массой тележек.
\end{itemize}

\section*{Исходные данные и рабочие формулы}
Для задания 1:
\begin{align*}
    m_1 &- \text{масса первой тележки}, \\
    m_2 &- \text{масса второй тележки}, \\
    v_{10} &- \text{скорость первой тележки до соударения}, \\
    v_{1x}, v_{2x} &- \text{скорости тележек после соударения}, \\
    p_{10x}, p_{1x}, p_{2x} &- \text{импульсы тел до и после соударения}, \\
    \delta p &- \text{относительное изменение импульса}, \\
    \delta W &- \text{относительное изменение кинетической энергии}.
\end{align*}

Основные формулы для расчета:

\begin{itemize}
    \item Импульс тела до соударения:
    \[
    p_{10x} = m_1 \cdot v_{10}
    \]
    \item Импульс тележек после соударения:
    \[
    p_{1x} = m_1 \cdot v_{1x}, \quad p_{2x} = m_2 \cdot v_{2x}
    \]
    \item Относительное изменение импульса:
    \[
    \delta p = \frac{p_{1x} + p_{2x} - p_{10x}}{p_{10x}}
    \]
    \item Относительное изменение кинетической энергии:
    \[
    \delta W = \frac{\frac{1}{2} m_1 v_{1x}^2 + \frac{1}{2} m_2 v_{2x}^2 - \frac{1}{2} m_1 v_{10}^2}{\frac{1}{2} m_1 v_{10}^2}
    \]
\end{itemize}

Для задания 2:
\begin{align*}
    m &- \text{масса гирьки}, \\
    v_1, v_2 &- \text{скорости тележки на разных участках}, \\
    a &- \text{ускорение тележки}, \\
    T &- \text{сила натяжения нити}.
\end{align*}

Основные формулы для задания 2:
\begin{itemize}
    \item Ускорение тележки:
    \[
    a = \frac{v_2 - v_1}{t}
    \]
    \item Сила натяжения нити:
    \[
    T = m \cdot a
    \]
\end{itemize}

\section*{Таблицы результатов}
Таблица 1.1: Измерения для задания 1
\begin{tabular}{|c|c|c|c|c|c|}
    \hline
    № опыта & $m_1$ (г) & $m_2$ (г) & $v_{10x}$ (м/с) & $v_{1x}$ (м/с) & $v_{2x}$ (м/с) \\
    \hline
    1 &  &  &  &  &  \\
    2 &  &  &  &  &  \\
    \hline
\end{tabular}

Таблица 3.1: Данные для экспериментов с гирьками
\begin{tabular}{|c|c|c|c|}
    \hline
    № опыта & Состав гирьки & $v_1$ (м/с) & $v_2$ (м/с) \\
    \hline
    1 & Подвеска &  &  \\
    2 & Подвеска + одна шайба &  &  \\
    \hline
\end{tabular}

\section*{Выводы}
\begin{itemize}
    \item Теоретическое значение относительного изменения энергии $\delta W$ соответствует экспериментальным данным.
    \item Табличные значения масс тележек согласуются с доверительными интервалами.
\end{itemize}

\end{document}

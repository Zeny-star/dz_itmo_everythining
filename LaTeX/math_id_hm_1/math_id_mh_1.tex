
\documentclass{report}

\usepackage[T2A]{fontenc}
\usepackage[russian]{babel}
\usepackage{graphicx}
\usepackage{float}
\usepackage{hyperref}
\usepackage{amsmath}
\usepackage{diffcoeff,amssymb}
\usepackage{mathtools}
\usepackage[normalem]{ulem}

\input{preamble}
\input{macros}
\input{letterfonts}



\title{\Huge{Матан инд}\\ Вариант №19}

\author{\huge{Евгений Турчанин}}
\date{}
\begin{document}
\maketitle

\qs{}{
Доказать что:
$\dfrac{1^4}{1\cdot3}+\dfrac{2^4}{3\cdot5}+\mathellipsis+\dfrac{n^4}{(2n-1)\cdot(2n+1)}=\dfrac{n(n+1)(n^2+n+1)}{6(2n+1)}$
}
\sol
Докажем через индукцию:\\
\begin{enumerate}
	\item Покажем что для $n=1$ верно: \\ 
		\begin{equation*}
		\dfrac{1(1+1)(1^2+1+1)}{6(2+1)}=\dfrac{1}{3}
		\end{equation*}
		\begin{equation*}
		\dfrac{1^4}{1\cdot3}=\dfrac{1}{3}
		\end{equation*}
	\item Пусть верно для $n$ \\
	\item Тогда докажем, что верно для $n+1$:
		\begin{equation*}
			\underbrace{\dfrac{1^4}{1\cdot3}+\dfrac{2^4}{3\cdot5}+\mathellipsis+\dfrac{n^4}{(2n-1)\cdot(2n+1)}}_{\dfrac{n(n+1)(n^2+n+1)}{6(2n+1)}}+\dfrac{(n+1)^4}{(2n+1)\cdot(2n+3)} \Rightarrow
		\end{equation*}
		Нужно доказать, что:
		\begin{equation*}
		\dfrac{n(n+1)(n^2+n+1)}{6(2n+1)}+\dfrac{(n+1)^4}{(2n+1)\cdot(2n+3)} = 
		\dfrac{(n+1)(n+2)(n^2+3n+3)}{6(2n+3)}
		\end{equation*}
		\sout{Трудно не заметить, что так оно и есть}
		Давайте покажем, что это так:\\
		Сократим все на $n+1$ и приведем к общему знаменателю:
		\begin{equation*}
		\dfrac{n(2n+3)(n^2+n+1)}{6(2n+1)(2n+3)}+\dfrac{6(n+1)^3}{6(2n+1)\cdot(2n+3)} = 
		\dfrac{(2n+1)(n+2)(n^2+3n+3)}{6(2n+1)(2n+3)}
		\end{equation*}
		Сократим на знаменатели и раскроем скобки:
		\begin{equation*}
		2x^4+11x^3+23x^2+21x+6=2x^3+11x^2+23x+21+6 \Rightarrow
		\end{equation*}
		\begin{equation*}
		0=0
		\end{equation*}
		\begin{center}
			Ч.Т.Д.
		\end{center}

\end{enumerate}



\qs{}{
Доказать что:
\[ \sum_{k=1}^{k}\dfrac{1}{k^2} \leqslant 2 -\dfrac{1}{n}\]
}
\sol
Докажем опять \sout{двадцатьпять} через индукцию:\\
\begin{enumerate}
	\item Покажем что для $n=1$ верно: \\ 
		\begin{equation*}
			\dfrac{1}{1^2}=1
		\end{equation*}
		\begin{equation*}
			2-\dfrac{1}{1}=1
		\end{equation*}
	\item Пусть верно для $n$ \\
	\item Тогда докажем, что верно для $n+1$:\\
		Те нужно доказать что верно:
		\begin{equation*}
			\underbrace{\dfrac{1}{1^2}+\dfrac{1}{2^2}+\mathellipsis+\dfrac{1}{n^2}}_{\leqslant 2-\frac{1}{n}}+\dfrac{1}{(n+1)^2} \leqslant 2-\dfrac{1}{n+1}
		\end{equation*}
		Приведем к общему знаменателю:
		\begin{equation*}
			\dfrac{(2n-1)(n+1)^2+n}{(n+1)^2n} \leqslant \dfrac{(2n+1)(n+1)n}{(n+1)^2n}
		\end{equation*}
		Сократим на знаменатель:
		\begin{equation*}
			(2n-1)(n+1)^2+n\leqslant (2n+1)(n+1)n
		\end{equation*}
		Раскроем скобки:
		\begin{equation*}
			2n^3+4n^2+2n-n^2-2n-1+n\leqslant 2n^3+2n^2+n^2+n \Rightarrow
		\end{equation*}
		\begin{equation*}
			-1\leqslant 0
		\end{equation*}
		\begin{center}
			Ч.Т.Д.
		\end{center}

\end{enumerate}


\qs{}{
$
C_n^0C_{2n}^n+C_n^1C_{2n}^{n-1}+C_n^2C_{2n}^{n-2}\mathellipsis C_n^nC_{2n}^0=C_{3n}^n
$
}
\sol Докажем по индукции:
\begin{enumerate}
	\item Покажем что для $n=1$ верно:\\
		\begin{equation*}
			C_1^0C_2^1+C_1^1C_2^0=1\cdot2+1=3
		\end{equation*}
		\begin{equation*}
			C_3^1=3
		\end{equation*}
	Ну, показал получается
	\item Пусть верно для $n$\\
	\item Тогда докажем, что верно для $n+1$:\\
		Нужно показать, что верно:
		\begin{equation*}
			C_{n+1}^0C_{2n+2}^{n+1} + C_{n+1}^1C_{2n+2}^n\mathellipsis C_{n+1}^{n+1}C_{2n+2}^0=C_{3n+3}^{n+1}
		\end{equation*}
		Перепишем эти выражения ввиде сумм:
		$\sum_{k=0}^{n}C_{2n+2}^k$

\end{enumerate}

\end{document}

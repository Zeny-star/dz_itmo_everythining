\documentclass{report}

\usepackage[T2A]{fontenc}
\usepackage[russian]{babel}
\usepackage{graphicx}
\usepackage{float}
\usepackage{hyperref}
\usepackage{amsmath}
\usepackage{diffcoeff,amssymb}


\input{preamble}
\input{macros}
\input{letterfonts}



\title{\Huge{Физика}\\ Решение Овчинкина}
\author{\huge{Евгений Турчанин}}
\date{}

\begin{document}

\maketitle
\dfn{2.6}{Найти ускорения $a_1$ и $a_2$ масс $m_1$ и $m_2$ и натяжение нити $T$ в системе, изображенной на рис.17. Массой блоков и нитей пренебречь.}
\sol Не теряя общности пусть ускорение $a_2$ направленно вниз, а ускорение $a_1$ соответственно вверх. Введем ось $y$ направленную вверх, тогда из 2 закона Ньютона на ось y:
\begin{equation}
\begin{cases}
	m_1 a_1=T-m_1g\\
	-m_2 a_2=2T-m_2g
\end{cases}
\end{equation}

Найдем кинематическую связь ускорений: Пусть $m_1$ сместился на $\Delta x$ вверх $\Rightarrow$ $m_2$ опустился на $\Delta x/2$ вниз, продифференцируем по времени 2 раза и найдем что:
\begin{center}
$a_1=-2a_2$
\end{center}

Используя эти уравнения находим что:
\begin{center}
$a_1$ = $\dfrac{2(m_2-2m_1)}{4m_1+m_2}g$; $a_2$ = $\dfrac{2(m_2-2m_1)}{4m_1+m_2}g$; $T$ = $\dfrac{3m_1m_2g}{4m_1+m_2}$
\end{center}
\clm{}{}{\bf{Ответ: }\\ \begin{center}$a_1$ = $\dfrac{2(m_2-2m_1)}{4m_1+m_2g}$; $a_2$ = $\dfrac{2(m_2-2m_1)}{4m_1+m_2}g$; $T$ = $\dfrac{3m_1m_2g}{4m_1+m_2}$ \end{center}}


\dfn{2.8}{Три груза висят на блоках (рис. 19). Крайние блоки неподвижны, а средний может передвигаться. Считая заданными $m_1$ и $m_2$, определить массу груза $m_3$, при которой средний блок будет оставаться неподвижным. Трением и массами блоков и веревки пренебречь.}


\sol Не теряя общности пусть ускорение $a_1$ направленно вверх, $a_2$ соответственно вниз. Введем ось $y$ направленную вверх, тк 3-й груз не двигается $\Rightarrow$ $a_1$ =$-a_2$=$a$, тогда из 2 закона Ньютона на ось y:
\begin{equation}
\begin{cases}
	m_1 a=T-m_1g\\
	-m_2 a=T-m_2g\\
	0=2T-m_3g
\end{cases}
\end{equation}
Из данной системы уравнений находим что $m_3$=$\dfrac{4m_1m_2}{m_1+m_2}$



\clm{}{}{\bf{Ответ:} \\ $m_3$ = $\dfrac{4m_1m_2}{m_1+m_2}$}


\dfn{2.12}{Камень массой $M$ лежит на горизонтальной плоскости на расстоянии $L$ от края пропасти. К камню прикреплена веревка, перекинутая через гладкий уступ; по веревке лезет обезьяна массой $m$. С каким постоянным(относительно земли) ускорением она должна лезть, чтобы успеть подняться раньше, чем упадет камень? Начальное расстояние обезьяны от уступа равно $H<(M/m)L$. Коэффициент трения камня о плоскость равен k. }


\sol Пусть ускорение бруска равно $a_1$, а ускорение макаки $a$, тогда введем оси $y$ и $x$, такие что ось $y$ направлена вверх, а $x \perp y$ и и направлена вправо. Тогда система уравнений: 

\begin{equation}
\begin{cases}
	Ma_1=T-Mgk  \quad \text{(для бруска на ось x)}\\
	T=m(g+a)  \quad \text{(для примата на ось y)}\\
\end{cases}
\end{equation}
Из этих уравнений: $a_1$=$\dfrac{m(g+a)-Mgk}{M}$\\
Запишем условие на то, что животное успеет подняться до уступа:
\begin{center}
\begin{equation}
\begin{cases}
L=a_1t_1^2/2\\
H=at_2^2/2\\
t_2<t_1
\end{cases}
\end{equation}
\end{center}
Тогда:
\begin{center}
\begin{equation}
	a>\frac{H}{L}a_1
\end{equation}
\end{center}

Подставляя $a_1$ в это выражение и получаем что: 

\begin{center}
\begin{equation}
	MLa>Hmg+Hma-MgHk \Rightarrow a>\dfrac{Hg(m-Mk)}{ML-Hm}
\end{equation}
\end{center}

\clm{}{}{\bf{Ответ:} \\ $a>\frac{Hg(m-Mk)}{ML-Hm}$}
P.S. Это ответ для $a$ когда, оно ускрорение ОТН веревки.

\dfn{2.18}{На столе лежит доска массой $M$ = 1 кг, а на доске - груз массой $m$ = 2кг. Какую силу $F$ нужно приложить к доске, чтобы доска выскользнула из-под груза? Коэффициент трения между грузом и доской 0.25, а между доской и столом - 0.5.}


\sol Пусть $\mu_1$ - коэффициент трения между доской и грузом, а $\mu_2$ - коэффициент трения между доской и столом, введем ось $x$, которая направлена вправо, рассмотрим критический случай, те когда $F_{тр}$ перестает быть силой трения покоя, тогда запишем 2 закон Ньютона на ось $x$:

\begin{equation}
\begin{cases}
	F_{тр}=mg\mu_1  \quad \text{(условие отрыва)}\\
F-(m+M)g\mu_2-mg\mu_1=Ma  \quad \text{(для доски)}\\
	ma=mg\mu_1  \quad $(для груза)$\\
\end{cases}
\end{equation}
Решая эту систему, получим $F=(m+M)(\mu_1+\mu_2)g=22.5 \Rightarrow$ $F \geqslant22.5$

\clm{}{}{\bf{Ответ:} \\ $F \geqslant22.5$}





\dfn{2.22}{По наклонной плоскости с углом наклона $\alpha$ соскальзывает брусок массой $m_1$, на котором находится второй брусок массой $m_2$. Коэффициент трения нижнего бруска о наклонную плоскость равен $k_1$, а коэффициент трения между брусками равен $k_2$, причем $k_1 > k_2$. Определить, будет ли двигаться верхний брусок относительно нижнего и каковы ускорения обоих брусков. Как изменится результат, если $k_1 < k_2 < \tg \alpha$?} \sol Рассмотрим два случая: \\1) Случай, когда $k_1 > k_2$: \\Введем систему координат, где ось x направлена вдоль наклонной плоскости вниз. Запишем уравнения движения для обоих брусков: \begin{equation} \begin{cases}
m_1a_1 = m_1g\sin\alpha - F_{тр1} - F_{тр2} \quad \text{(для нижнего бруска)} \\
m_2a_2 = m_2g\sin\alpha - F_{тр2} \quad \text{(для верхнего бруска)}
\end{cases}
\end{equation}

Где $F_1$ - сила трения между нижним и верхним бруском, а $F_2$ - сила трения между нижним бруском и полом.\\

Решая эту систему, получим:

\begin{equation}
\begin{cases}
a_1 = g(\sin\alpha - k_1\cos\alpha) - \frac{m_2}{m_1}g(k_2-k_1)\cos\alpha \\
a_2 = g(\sin\alpha - k_2\cos\alpha)
\end{cases}
\end{equation}\\

2) Случай, когда $k_1 < k_2 < \tg \alpha$:

В этом случае верхний брусок не будет двигаться относительно нижнего, тогда :

\begin{equation}
(m_1 + m_2)a = (m_1 + m_2)g\sin\alpha - k_1(m_1 + m_2)g\cos\alpha
\end{equation}

Отсюда ускорение обоих брусков:

\begin{equation}
a = g(\sin\alpha - k_1\cos\alpha)
\end{equation}

\clm{}{}{\bf{Ответ:} \\ 
1) При $k_1 > k_2$: \\
   $a_1 = g(\sin\alpha - k_1\cos\alpha) - \frac{m_2}{m_1}g(k_2-k_1)\cos\alpha$ \\
   $a_2 = g(\sin\alpha - k_2\cos\alpha)$ \\
2) При $k_1 < k_2 < \tg \alpha$: \\
   $a_1 = a_2 = g(\sin\alpha - k_1\cos\alpha)$} 






\dfn{2.32}{Парусный буер масой 100кг начинает движение под действием ветра, дующего со скоростью $v$=10м/с. Вычислить время, через которое мощность, отбираемая буером у ветра, будет максимальной, если сила сопротивления паруса ветру пропорциональна квадрату относительной скорости между буером и ветром с коэффициентом пропорциональности $k$=0,1кг/м. Трением пренебречь.}


\sol Из условия: 
\begin{center}
\begin{equation}
	F=kv_{reg}^2 \Rightarrow F=k(v-u)^2
\end{equation}
\end{center}
- где $v$ - скорость ветра, $u$ - скорость буера\\
По определению: 
\begin{center}
\begin{equation}
P=F*u=k(v-u)^2u 
\end{equation}
\end{center}
Возьмем от данного выражения производную по $u$ и приравняем к нулю:
\begin{center}
\begin{equation}
(k(v-u)^2u)' = 0 \Rightarrow kv^2-4kvu+4ku^2=0 \Rightarrow v=3u
\end{equation}
\end{center}

Зная это отношение, найдем время $t$:

\begin{center}
\begin{equation}
	-m \diff{u}{t}=k(v-u)^2 \Rightarrow -\frac{m}{k} \int_0^{v/3} \frac{\dl u}{(v-u)^2}=\int \dl t
	\Rightarrow
\end{equation}
\begin{equation}
	\Rightarrow \int_0^{v/3} \frac{\dl (v-u)}{(v-u)^2}=\frac{m}{k}
	\left(\frac{1}{v-v/3}-\frac{1}{v}\right) \Rightarrow t=\dfrac{m}{2kv}
\end{equation}
\end{center}

\clm{}{}{\bf{Ответ:} \\
$t = \frac{m}{2kv}$=50c}


\dfn{2.38}{Тело бросают вертикально вверх в вязкой среде. Сила вязкого трения пропорциональна скорости движения тела. Вычислить время $t_1$ подъема(спуска) тела на максимальную высоту его полета вверх и сравнить его со временем $t_0$ подъема в отсутствие трения. Начальная скорость тела в обоих случаях одинакова.}


\sol Запишем уравнение движения: 

\begin{center}
\begin{equation}
-ma=mg+\alpha v \Rightarrow -\diff{v}{t}=g+ \frac{\alpha v}{m} \Rightarrow 
\end{equation}
\begin{equation}
\Rightarrow -\int \dfrac{\dl v}{g+\frac{\alpha v}{m}}=\int \dl t
\end{equation}
\end{center}


Поделим, и домножим на$\frac{\alpha}{m}$, и занесем g под 
$\dl v:$ \begin{center}
\begin{equation}
 -\frac{m}{\alpha} \int \dfrac{\dl{ (\frac{\alpha v}{m}+g})}{g+\frac{\alpha v}{m}}=\int \dl t \Rightarrow t_1=\frac{m}{\alpha}
\ln \left(1+\dfrac{\alpha v_0}{mg}\right)
\end{equation}
\end{center}


\clm{}{}{\bf{Ответ:}\\ $t_1=\frac{m}{\alpha}
\ln \left(1+\dfrac{\alpha v_0}{mg}\right)$}

\end{document}

\documentclass[a4paper]{article}

\usepackage[T2A]{fontenc}
\usepackage[russian]{babel}
\usepackage{graphicx}
\usepackage{float}
\usepackage{hyperref}
\usepackage{amsmath, amssymb}
\usepackage{caption}
\usepackage{geometry}
\usepackage{pdfpages}
\usepackage{siunitx}
\geometry{top=2cm,bottom=2cm,left=2cm,right=2cm}

\newcommand{\minus}{\scalebox{0.75}[1.0]{$-$}}




\begin{document}

\begin{center}
\textsc{Санкт-Петербургский национальный исследовательский институт информационных технологий, механики и оптики\\[3mm]
Физический факультет} \\[3mm]

\end{center}
\vspace{5mm}
\line(1,0){\textwidth}
\begin{center}
\textbf{ЛАБОРАТОРНАЯ РАБОТА №1.13\\}
\textbf{"Изучение прецессии гироскопа"}
\end{center}
\vspace{2mm}
\line(1,0){\textwidth}
\vspace{5mm}
\begin{minipage}{0.4\textwidth}
    Группа: Z3144 \\
    Студент: Евгений Турчанин\\
    \vspace{1mm}
\end{minipage}
\hfill
\vspace{1mm}
\line(1,0){\textwidth}



\section{\textbf{Цели работы}}
\begin{enumerate}
\item Наблюдение прецессии гироскопа.
\item Экспериментальное подтверждение линейно зависимости периода прецессии гироскопа от частоты вращения гироскопа вокруг оси симметрии.
\item Экспериментальное определение момента инерции гироскопа
\end{enumerate}



\section{\textbf{Задачи}}
\begin{enumerate}
\item Измерить период прецессии гироскопа.
\item Измерить частоту вращения гироскопа вокруг своей оси.
\item Рассчитать момент инерции гироскопа относительно оси вращения используя данные полученные в ходе экперимента. Сравнить полученный результат с моментом инерции гироскопа, рассчитанным теоретически.
\end{enumerate}

\section{\textbf{Теория}}

Рассмотрим гироскоп закреплённый в точке совпадающей с центром масс так, что ось гироскопа, лежащая в горизонтальной
плоскости, может свободно поворачиваться в любом направлении. Пусть угловая скорость $\omega$ совпадает по направлению с осью вращение гироскопа, т.е. полный момент импульса:
\begin{equation}
\mathbf{M} = \mathbf{I} \omega
\end{equation}

где $I$ – момент инерции гироскопа относительно оси вращения,
совпадающей с одной из главных центральных осей. Пусть к оси
гироскопа приложена некоторая постоянная внешняя сила $F$, как
это показано на (рис. 1), т.е. перпендикулярно оси гироскопа.

\begin{figure}[H]
\begin{center}
\includegraphics[width=0.3\textwidth]{pick_1.png}
\end{center}
\end{figure}

На ось гироскопа действует момент внешних сил $L$, по модулю равный:
\begin{equation}
L=Fl
\end{equation}

где $l$ – плечё силы $F$. Из уравнения моментов можно определить
направление вращения оси гироскопа:
\begin{equation}
    d \mathbf{M}=\mathbf{L} d t
\end{equation}
которая вращается с некоторой постоянной угловой скоростью $\Omega$,
называемой угловой скоростью прецессии, вокруг вертикальной
оси, проходящей через точку опоры гироскопа. Получим формулу, связывающую угловую скорость прецессии с угловой скоростью вращения гироскопа.\\
Пусть $d \varphi$ – угол на который поворачивается ось гироскопа вокруг вертикальной оси за время $d t$, тогда по определению:

\begin{equation}
    \Omega = \frac{\varphi}{d t}
\end{equation}
Модуль изменения момента импульс при этом можно записать
как,
\begin{equation}
    d M = M d \varphi
\end{equation}
с учётом формул (1), (2), (3), (4), (5) получим, что:
\begin{equation}
\Omega = \frac{Fl}{I\omega}
\end{equation}

т.о. зависимость угловой скорости прецессии от угловой скорости вращения гироскопа является обратно пропорциональной. Поскольку на эксперименте, чаще всего удобнее измерять период
нутации, а не угловую скорость, то формулу (7) удобно переписать в виде:
\begin{equation}
    T'=\dfrac{2\pi}{Fl}I\omega
\end{equation}
где $T'$ – период прецессии.


\section{\textbf{Данные}}

\begin{figure}[H]
\begin{center}
\includegraphics[width=0.3\textwidth]{pick_2.png}
\includegraphics[width=0.5\textwidth]{pick_3.png}
\end{center}
\end{figure}


\section{\textbf{Результаты}}

С помощью питона обрабатываем данные и получаем следующие результаты:


\begin{figure}[H]
\begin{center}
\includegraphics[width=0.7\textwidth]{pick_4.png}
\end{center}
\end{figure}


\[
    I_{\text{exp, 1}} = 0.01179 \pm 0.00565 \, , \quad \text{или} \quad [0.00614, 0.01744] \,\text{кг$\cdot$ м$^2$}
\]

\[
I_{\text{exp, 2}} = 0.01693 \pm 0.00322, \quad \text{или} \quad [0.01371, 0.02015] \,\text{кг$\cdot$ м$^2$}
\]

\[
I_{\text{exp, 3}} = 0.01472 \pm 0.00789, \quad \text{или} \quad [0.00683, 0.02261] \,\text{кг$\cdot$ м$^2$}
\]
\[
\Delta I_{\text{1}} = |0.01179 - 0.01172| = 0.00007 \,\text{кг$\cdot$ м$^2$}
\]

\[
\Delta I_{\text{2}} = |0.01693 - 0.01172| = 0.00521 \,\text{кг$\cdot$ м$^2$}
\]

\[
\Delta I_{\text{3}} = |0.01472 - 0.01172| = 0.00300 \,\text{кг$\cdot$ м$^2$}
\]
\section{\textbf{Вывод}}
Из результатов можно сделать вывод, что теория совпадает с экспериментальными данными. Малые отклонения могут быть вызваны несколькими причинами:
\begin{enumerate}
	\item Время замерялось ручным таймером, те есть человеческая реакция
	\item Во время эксперимента регулировочный груз менял свое положение, поэтому за период главная ось отклонялась на малый угол
	\item При движении гироскопа, есть небольшая нутация, которая приводит к изменению времени движения
\end{enumerate}
\end{document}

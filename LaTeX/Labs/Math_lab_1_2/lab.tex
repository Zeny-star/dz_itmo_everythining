\documentclass{report}

\usepackage[T2A]{fontenc}
\usepackage[russian]{babel}
\usepackage{graphicx}
\usepackage{float}
\usepackage{hyperref}
\usepackage{amsmath}
\usepackage{diffcoeff,amssymb}
\usepackage{mathtools}
\usepackage[normalem]{ulem}

\input{preamble}
\input{macros}
\input{letterfonts}

\setcounter{secnumdepth}{0}
\title{\Huge{Матан Лаба}}
\author{\huge{Павел Андреев, Григорий Горбушкин, Евгений Турчанин}}
\date{}
\begin{document}
\maketitle


\section{\textbf{Теория}}
\qs{}{
Объяснить, почему разность назад $f_-(x_0)$ – разумное приближение производной в точке $x_0$.
}
\begin{equation}
  f_-(x_0)=\dfrac{f(x_0)-f(x_0-h)}{h}\text{,}
\end{equation}

разложим $f(x_0-h)$ в ряд Тейлора:

\begin{equation}
  f(x_0-h)=f(x_0)-hf'(x_0)+o(h)\text{,}
\end{equation}
подставим (2) в (1)
\begin{equation}
f_-(x_0)=\dfrac{hf'(x_0)-o(h))}{h}=f'(x_0)-o(1)
\end{equation}
Тк их разница равна $o(1)$, то $f_-(x_0)$ – разумное приближение производной в точке $x_0$.

\qs{}{
Формально показать, как получаются формулы для оценки погрешности в случае приближения производной первой (односторонней) разностью.
}

\begin{equation}
  f_-(x_0)=\dfrac{f(x_0)-f(x_0-h)}{h}\text{,}
\end{equation}

разложим $f(x_0-h)$ в ряд Тейлора:

\begin{equation}
  f(x_0-h)=f(x_0)-hf'(x_0)+f''(\xi_1)\dfrac{h^2}{2}\text{,}
\end{equation}
где $\xi_1 \in (x_0-h,x_0)$, подставим (5) в (4)
\begin{equation}
  f_-(x_0)=\dfrac{hf'(x_0)-f''(\xi_1))\dfrac{h^2}{2}}{h}=f'(x_0)-f''(\xi_1)\dfrac{h}{2}
\end{equation}

Пусть $M=\sup \{f''(\xi_1)\}$, тогда разность можно оценить как:

\begin{equation}
|f_-(x_0)-f'(x_0)|\leq M\dfrac{h}{2}\text{.}
\end{equation}
Тк ошибка линейно зависит от $h$, следовательно мы можем сделать ее сколь угодно малой, тк $M$ --- конечное число

\qs{}{
Объяснить, почему центральная разность $f_{\circ 1}(x_0)$ – разумное приближение производной в точке $x_0$.
}
\begin{equation}
  f_{\circ 1}(x_0)=\dfrac{f(x_0+h)-f(x_0-h)}{2h}
\end{equation}
Разложим $f(x_0+h)$ и $f(x_0-h)$ в ряд Тейлора:

\begin{equation}
  f(x_0+h)=f(x_0)+hf'(x_0)+\dfrac{h^2}{2}f''(x_0)+o(h^2)\text{,}
\end{equation}
\begin{equation}
f(x_0-h)=f(x_0)-hf'(x_0)+\dfrac{h^2}{2}f''(x_0)+o(h^2)\text{,}
\end{equation}
подставим (10) и (9) в (8)
\begin{equation}
  f_{\circ 1}(x_0)=\dfrac{2hf'(x_0)+h^2f''(x_0)+o(h^2)}{2h}=f'(x_0)+\dfrac{h}{2}f''(x_0)+o(h)\text{,}
\end{equation}
Понятно, что при $h \to 0$, $f_{\circ 1}(x_0)$ - $f'(x_0)$ идет к 0

\qs{}{
Формально показать, как получаются формулы для оценки погрешности в случае приближения производной центральной разностью.
}
Опять же разложим в ряд Тейлора:
\begin{equation}
  f(x_0+h)=f(x_0)+hf'(x_0)+\dfrac{h^2}{2}f''(x_0)+f'''(\xi_1)\dfrac{h^3}{6}\text{,}
\end{equation}
\begin{equation}
  f(x_0-h)=f(x_0)-hf'(x_0)+\dfrac{h^2}{2}f''(x_0)+f'''(\xi_2)\dfrac{h^3}{6}\text{,}
\end{equation}

Пусть $M=\sup \{f'''(\xi_1)+ f'''(\xi_2)\}$, тогда разность можно оценить как:
\begin{equation}
  |f_{\circ 1}(x_0)-f'(x_0)|\leq M\dfrac{h^2}{12}\text{.}
\end{equation}

\qs{}{
Узнать, как хранятся числа, скажем, в python. Узнать, что такое машинная точность. Объяснить, почему в python 0.1+0.2 != 0.3.
}
Число в python хранится в виде $\pm m\cdot2^s$, где $m,s \in \mathbb{N}$, а 0.1, 0.2 и 0.3, в этой системе --- это бесконечные переодические дроби в этой системе\\
Машинная точность --- это такое минимальное число $\varepsilon$, что $1+\varepsilon \neq 1$, она примерно равна $2^{-52}$\\
Ошибка округления получается, тк $f(x_0+h)$ и $f(x_0)$ --- округлены до машинной точности, а делятся на $h$, те р
ост ошибки $O(1/h)$
\qs{}{
\begin{enumerate}
  \item Выбрать любую дважды дифференцируемую функцию $f$.
  \item Аппроксимировать производную с помощью разностей $f_{\pm}(x_0)$, $f_{\circ 1}(x_0)$.
  \item Построить график зависимости $\Delta(x, \varepsilon)$ от $h$, $h \in (10^{-20},\ 1)$, \[\Delta(x, h)=\left| f'(x)-f_{\pm, \circ_1}\right|\text{,} \] при разных $x$ (шкала по оси $y$ – логарифмическая!).
  \item Проинтерпретировать полученный результат.
\end{enumerate}
}

Пусть наша \sout{дорогая} функция $f(x)=x^3$, тогда аппроксимация производной


\begin{equation}
  f_+(x_0)=\dfrac{(x_0+h)^3-(x_0)^3}{h}=3x^2+3hx+h^2
\end{equation}
\begin{equation}
f_-(x_0)=\dfrac{(x_0)^3-(x_0-h)^3}{h}=3x^2-3hx+h^2
\end{equation}
\begin{equation}
  f_{\circ 1}(x_0)=\dfrac{(x_0+h)^3-(x_0-h)^3}{2h}=3x^2+h^2
\end{equation}

\begin{figure}[H]
  \centering
  \includegraphics[scale=0.5]{fig_1.png}
  \includegraphics[scale=0.5]{fig_2.png}
  \includegraphics[scale=0.5]{fig_3.png}
\end{figure}

Пусть наша функция $f(x)=x^2\sin\left(\frac{1}{x}\right)$ если $x\neq 0$ и $0$, если $x=0$. Понятно что в 0 она не дифф-ма 2-й раз

\begin{equation}
  f_+(0)=\dfrac{h^2\sin\frac{1}{h}}{h}=h\sin\frac{1}{h}\quad h\to0
\end{equation}

\begin{equation}
  f_-(0)=\dfrac{(-h)^2\sin\frac{1}{-h}}{-h}=-h\sin\frac{1}{-h}\quad h\to0
\end{equation}
\[
  f_{\circ 1}(x)=\dfrac{(h+x)^2\sin\frac{1}{x+h}-(x-h)^2\sin\frac{1}{x-h}}{2h}=
\]
\[
  -\dfrac{1}{2}h\sin\left(\frac{1}{x-h}\right)+x\sin\left(\frac{1}{x-h}\right)-\frac{x^2\sin\left(\frac{1}{x-h}\right)}{2h}+\dfrac{1}{2}h\sin\left(\frac{1}{h+x}\right)+x\sin\left(\frac{1}{x+h}\right)+\dfrac{x^2\sin\left(\frac{1}{x+h}\right)}{2h} \quad x\to0; \; h \to 0
\]

\begin{figure}[H]
\centering
\includegraphics[scale=0.4]{fig_4.png}
\includegraphics[scale=0.4]{fig_5.png}
\includegraphics[scale=0.4]{fig_6.png}
\end{figure}

\section{\textbf{\sout{Экскримент} Эксперимент}}

Используя программу tracker, получаем $\theta_0=44.59^{\circ}$.

\begin{enumerate}
\item
Разностью вперед:

\begin{equation}
  \delta t_{\text{верх}} = (1 - \dfrac{0.205}{0.211}) \cdot 100 \% \approx 2.8\% \text{,}
\end{equation}
\begin{equation}
  \delta t_{\text{пад}} = (1 - \dfrac{0.41}{0.421}) \cdot 100 \% \approx 2.6\% \text{,}
\end{equation}

\begin{equation}
  \delta x_{\text{верх}} = (1 - \dfrac{0.623}{0.64}) \cdot 100 \% \approx 2.7\% \text{,} 
\end{equation}

\item Разностью назад:


\begin{equation}
  \delta t_{\text{верх}} = (1 - \dfrac{0.211}{0.228}) \cdot 100 \% \approx 7.5\% \text{,}
\end{equation}

\begin{equation}
  \delta t_{\text{пад}} = (1 - \dfrac{0.4096}{0.4561}) \cdot 100 \% \approx 10.2\% \text{,}
\end{equation}

\begin{equation}
  \delta x_{\text{верх}} = (1 - \dfrac{0.6231}{0.7202}) \cdot 100 \% \approx 13.5\% \text{,}
\end{equation}


\item Средним значением:

\begin{equation}
  \delta t_{\text{верх}} = (1 - \dfrac{0.2109}{0.2193}) \cdot 100 \% \approx 3.9\% \text{,}
\end{equation}

\begin{equation}
  \delta t_{\text{пад}} = (1 - \dfrac{0.4096}{0.4386}) \cdot 100 \% \approx 6.6\% \text{,}
\end{equation}

\begin{equation}
  \delta x_{\text{верх}} = (1 - \dfrac{0.6231}{0.6802}) \cdot 100 \% \approx 8.4\% \text{,}
\end{equation}

\begin{figure}[H]
\center
\includegraphics[scale=0.6]{gra_1.png}
\end{figure}

\end{enumerate}

Для $\theta_0= 9.14^{\circ}$. Рассчитаем $v_0$ разными методами:

\begin{enumerate}
\item Разностью вперед:

\begin{equation}
\begin{aligned}
  v_{0x} &= \dfrac{0.131 - 0.046}{0.033} \approx 2.59\text{,} \quad
  v_{0y} &= \dfrac{0.725 - 0.716}{0.033} \approx 0.27\text{,}\\
\end{aligned}
\end{equation}
    \begin{equation}
  v_0 = \sqrt{v_{0x}^2 + v_{0y}^2} \approx 2.61\text{,}
    \end{equation}

\begin{equation}
  t_{\text{верх}} = \dfrac{v_0 \sin(\theta)}{g}\approx 0.042\text{,}
\end{equation}

\begin{equation}
  t_{\text{пад}} = \dfrac{v_0 \sin(\theta)+\sqrt{v_0 \sin(\theta)^2+2 g y_0}}{g} \approx 0.337\text{,}
\end{equation}

\begin{equation}
  x_{\text{пад}} = v_0\cdot\cos{\theta}\cdot t_{\text{пад}} \approx 0.868\text{,}
\end{equation}
Погрешности:

\begin{equation}
  \delta t_{\text{верх}} = (1 - \dfrac{0.042}{0.063}) \cdot 100 \% \approx 33.23\% \text{,}
\end{equation}

\begin{equation}
  \delta t_{\text{пад}} = (1 - \dfrac{0.337}{0.360}) \cdot 100 \% \approx 6.37\% \text{,}
\end{equation}

\begin{equation}
  \delta x_{\text{верх}} = (1 - \dfrac{0.868}{0.963}) \cdot 100 \% \approx 9.87\% \text{,}
\end{equation}

\item Разностью назад:

\begin{equation}
\begin{aligned}
  v_{0x} &= \dfrac{0.046 - (-0.039)}{0.033} \approx 2.58\text{,} \quad
  v_{0y} &= \dfrac{0.716 - 0.697}{0.033} \approx 0.56\text{,}\\
\end{aligned}
\end{equation}
    \begin{equation}
  v_0 = \sqrt{v_{0x}^2 + v_{0y}^2} \approx 2.64\text{,}
    \end{equation}

\begin{equation}
  t_{\text{верх}} = \dfrac{v_0 \sin(\theta)}{g}\approx 0.043\text{,}
\end{equation}

\begin{equation}
  t_{\text{пад}} = \dfrac{v_0 \sin(\theta)+\sqrt{v_0 \sin(\theta)^2+2 g y_0}}{g} \approx 0.337\text{,}
\end{equation}


\begin{equation}
  x_{\text{пад}} = v_0\cdot\cos{\theta}\cdot t_{\text{пад}} \approx 0.878\text{,}
\end{equation}
Погрешности:

\begin{equation}
  \delta t_{\text{верх}} = (1 - \dfrac{0.043}{0.063}) \cdot 100 \% \approx 33.42\% \text{,}
\end{equation}

\begin{equation}
  \delta t_{\text{пад}} = (1 - \dfrac{0.337}{0.360}) \cdot 100 \% \approx 6.50\% \text{,}
\end{equation}

\begin{equation}
  \delta x_{\text{верх}} = (1 - \dfrac{0.878}{0.963}) \cdot 100 \% \approx 8.89\% \text{,}
\end{equation}


\item Средним значением:

\begin{equation}
\begin{aligned}
  v_{0x} &= \dfrac{0.131 - (-0.039)}{0.066} \approx 2.59\text{,} \quad
  v_{0y} &= \dfrac{0.725 - 0.697}{0.066} \approx 0.42\text{,}\\
\end{aligned}
\end{equation}
    \begin{equation}
  v_0 = \sqrt{v_{0x}^2 + v_{0y}^2} \approx 2.62\text{,}
    \end{equation}

\begin{equation}
  t_{\text{верх}} = \dfrac{v_0 \sin(\theta)}{g}\approx 0.042\text{,}
\end{equation}

\begin{equation}
  t_{\text{пад}} = \dfrac{v_0 \sin(\theta)+\sqrt{v_0 \sin(\theta)^2+2 g y_0}}{g} \approx 0.337\text{,}
\end{equation}

\begin{equation}
  x_{\text{пад}} = v_0\cdot\cos{\theta}\cdot t_{\text{пад}} \approx 0.872\text{,}
\end{equation}
Погрешности:

\begin{equation}
  \delta t_{\text{верх}} = (1 - \dfrac{0.042}{0.063}) \cdot 100 \% \approx 32.92\% \text{,}
\end{equation}

\begin{equation}
  \delta t_{\text{пад}} = (1 - \dfrac{0.337}{0.360}) \cdot 100 \% \approx 6.42\% \text{,}
\end{equation}

\begin{equation}
  \delta x_{\text{верх}} = (1 - \dfrac{0.872}{0.963}) \cdot 100 \% \approx 9.50\% \text{,}
\end{equation}
\end{enumerate}

Для $\theta_0= 41.288^{\circ}$. Рассчитаем $v_0$ разными методами:

\begin{enumerate}
\item Разностью вперед:

\begin{equation}
\begin{aligned}
  v_{0x} \approx 2.394\text{,} \quad
  v_{0y} \approx 1.912\text{,}\\
\end{aligned}
\end{equation}
    \begin{equation}
  v_0 = \sqrt{v_{0x}^2 + v_{0y}^2} \approx 3.064\text{,}
    \end{equation}

\begin{equation}
  t_{\text{верх}} = \dfrac{v_0 \sin(\theta)}{g}\approx 0.206\text{,}
\end{equation}

\begin{equation}
  t_{\text{пад}} = \dfrac{-v_0 \sin(\theta)+\sqrt{v_0 \sin(\theta)^2+2 g y_0}}{g} \approx 0.452\text{,}
\end{equation}

\begin{equation}
  x_{\text{пад}} = v_0\cdot\cos{\theta}\cdot t_{\text{пад}} \approx 1.041 \text{,}
\end{equation}
Погрешности:

\begin{equation}
  \delta t_{\text{верх}} \approx 17.81\% \text{,}
\end{equation}

\begin{equation}
  \delta t_{\text{пад}}  \approx 9.58\% \text{,}
\end{equation}

\begin{equation}
  \delta x_{\text{верх}} \approx 11.52\% \text{,}
\end{equation}

\item Разностью назад:

\begin{equation}
\begin{aligned}
  v_{0x} \approx 2.342\text{,} \quad
  v_{0y} \approx 2.247\text{,}\\
\end{aligned}
\end{equation}
    \begin{equation}
  v_0 = \sqrt{v_{0x}^2 + v_{0y}^2} \approx 3.246\text{,}
    \end{equation}

\begin{equation}
  t_{\text{верх}} = \dfrac{v_0 \sin(\theta)}{g}\approx 0.218\text{,}
\end{equation}

\begin{equation}
  t_{\text{пад}} = \dfrac{v_0 \sin(\theta)+\sqrt{v_0 \sin(\theta)^2+2 g y_0}}{g} \approx 0.475\text{,}
\end{equation}

\begin{equation}
  x_{\text{пад}} = v_0\cdot\cos{\theta}\cdot t_{\text{пад}} \approx 1.157\text{,}
\end{equation}
Погрешности:

\begin{equation}
  \delta t_{\text{верх}}\approx 12.94\% \text{,}
\end{equation}

\begin{equation}
  \delta t_{\text{пад}} \approx 5.08\% \text{,}
\end{equation}

\begin{equation}
  \delta x_{\text{верх}} \approx 1.61\% \text{,}
\end{equation}


\item Средним значением:

\begin{equation}
\begin{aligned}
  v_{0x}\approx 2.37\text{,} \quad
  v_{0y} \approx 2.08\text{,}\\
\end{aligned}
\end{equation}
    \begin{equation}
  v_0 = \sqrt{v_{0x}^2 + v_{0y}^2} \approx 3.15\text{,}
    \end{equation}

\begin{equation}
  t_{\text{верх}} = \dfrac{v_0 \sin(\theta)}{g}\approx 0.218\text{,}
\end{equation}

\begin{equation}
  t_{\text{пад}} = \dfrac{v_0 \sin(\theta)+\sqrt{v_0 \sin(\theta)^2+2 g y_0}}{g} \approx 0.463\text{,}
\end{equation}

\begin{equation}
  x_{\text{пад}} = v_0\cdot\cos{\theta}\cdot t_{\text{пад}} \approx 1.096\text{,}
\end{equation}
Погрешности:

\begin{equation}
  \delta t_{\text{верх}}\approx 15.46\% \text{,}
\end{equation}

\begin{equation}
  \delta t_{\text{пад}} \approx 7.41\% \text{,}
\end{equation}

\begin{equation}
  \delta x_{\text{верх}} \approx 6.81\% \text{,}
\end{equation}


\end{enumerate}


\end{document}

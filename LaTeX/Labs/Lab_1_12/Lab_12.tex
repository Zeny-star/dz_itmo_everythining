\documentclass[a4paper]{article}

\usepackage[T2A]{fontenc}
\usepackage[russian]{babel}
\usepackage{graphicx}
\usepackage{float}
\usepackage{hyperref}
\usepackage{amsmath, amssymb}
\usepackage{caption}
\usepackage{geometry}
\usepackage{pdfpages}
\geometry{top=2cm,bottom=2cm,left=2cm,right=2cm}

\newcommand{\minus}{\scalebox{0.75}[1.0]{$-$}}



\begin{document}

\begin{center}
\textsc{Санкт-Петербургский национальный исследовательский институт информационных технологий, механики и оптики\\[3mm]
Физический факультет} \\[3mm]

\end{center}
\vspace{5mm}
\line(1,0){\textwidth}
\begin{center}
\textbf{""}
\end{center}
\vspace{2mm}
\line(1,0){\textwidth}
\vspace{5mm}
\begin{minipage}{0.4\textwidth}
    Группа: Z3144 \\
    Студент: Евгений Турчанин\\
    \vspace{1mm}
\end{minipage}
\hfill
\vspace{1mm}
\line(1,0){\textwidth}


\section{ \textbf{Цели работы}}

Изучение режимов колебаний в простейшей системе двух связанных осцилляторов и сопоставление с элементарной теорией связныхосцилляторов.

\section{\textbf{Задачи}}

\begin{enumerate}
    \item  Измерение частоты синфазной колебательной моды системы.
    \item  Измерение частоты при колебаниях системы в противофазе.
Измерение константы связи и коэффициента жёсткости пружины.
    \item  Измерение периода и частоты биений, возникающих при возбуждении двумодового колебательного процесса.
\end{enumerate}


\section{\textbf{Теоретическое введение}}

\begin{enumerate}
    \item a
\end{enumerate}





\end{document}

\documentclass{report}

\usepackage[T2A]{fontenc}
\usepackage[russian]{babel}
\usepackage{graphicx}
\usepackage{float}
\usepackage{hyperref}
\usepackage{amsmath}
\usepackage{diffcoeff,amssymb}
\usepackage{mathtools}
\usepackage[normalem]{ulem}

\input{preamble}
\input{macros}
\input{letterfonts}


\setcounter{secnumdepth}{0}
\title{\Huge{Матан Лаба}}
\author{\huge{Павел Андреев, Григорий Горбушкин, Евгений Турчанин}}
\date{}
\begin{document}
\maketitle

\section{\textbf{Вавилонский метод}}
\begin{enumerate}
	\item Покажем, что данная последовательность сходится куда надо.
	Предположим, что она действительно сходится, тогда пусть $A$ - то, куда она сходится, тогда:
	\begin{equation*}
	A=\frac{1}{2}\left(A+\dfrac{a}{A}\right)\Rightarrow A=\sqrt{a}
	\end{equation*}
	Покажем, что данная последовательность вообще сходится:
	\begin{equation*}
		x_n-x_{n-1}=\frac{1}{2}\left(x_{n-1}+\dfrac{a}{x_{n-1}}\right)-x_{n-1}=-\frac{1}{2}x_{n-1}+\frac{1}{2}\dfrac{a}{x_{n-1}}
	\end{equation*}
	Нужно показать больше или меньше нуля это выражение, для этого сравним два числа:
	\begin{equation*}
		\frac{1}{2}x_{n-1}\lessgtr\frac{1}{2}\dfrac{a}{x_{n-1}}\Rightarrow x_{n-1}^2\lessgtr a \Rightarrow \text{тк}\;x_{n-1}>\sqrt{a} \Rightarrow \text{Убывает}
	\end{equation*}
	Покажем в явном виде, что $x_{n-1}>\sqrt{a}$
	\begin{enumerate}
		\item Покажем для n=2:
			\begin{equation*}
				x_1=\frac{1}{2}\left(x_{0}+\dfrac{a}{x_{0}}\right) \Rightarrow
				x_2=\frac{1}{2}\left(x_{1}+\dfrac{a}{x_{1}}\right)=\frac{1}{2}\left(\frac{1}{2}\left(x_{0}+\dfrac{a}{x_{0}}\right)+\dfrac{a}{\dfrac{1}{2}\left(x_{0}+\dfrac{a}{x_{0}}\right)}\right)=
				\frac{1}{4}\left(x_0+\dfrac{a}{x_0}\right)+\dfrac{a}{\left(x_0+\dfrac{a}{x_0}\right)}
			\end{equation*}
			Теперь \sout{гениальный финт ушами} по неравенству Каши\sout{на}:
			\begin{equation*}
				\frac{1}{4}\left(x_0+\dfrac{a}{x_0}\right)+\dfrac{a}{\left(x_0+\dfrac{a}{x_0}\right)}\geq 2\sqrt{\frac{1}{4}\left(x_0+\dfrac{a}{x_0}\right)\cdot\dfrac{a}{\left(x_0+\dfrac{a}{x_0}\right)}}=a\Rightarrow x_2\geq \sqrt{a}
			\end{equation*}
		\item Пусть верно для n-1
		\item Покажем что верно для n:
			\begin{equation*}
				x_n=\frac{1}{2}\left(x_{n-1}+\dfrac{a}{x_{n-1}}\right) \geq 2\sqrt{\frac{1}{4}x_{n-1}\cdot\dfrac{a}{x_{n-1}}}=\sqrt{a}
			\end{equation*}

	\end{enumerate}
	Отсюда получаем и ограниченность \sout{И вообще кто молодец? Я молодец! Правильно? Правильно!}
\item Из выше написанного следует что $\dfrac{x_n}{\sqrt{a}}-1\geq 0$\\
	Покажем что $\varepsilon_{n+1}=\dfrac{\varepsilon_n^2}{2(1+\varepsilon_n)}$:
	\begin{equation*}
		\varepsilon_{n+1}=\dfrac{x_{n+1}}{\sqrt{a}}-1=\frac{1}{2\sqrt{a}}\left(x_{n}+\dfrac{a}{x_{n}}\right)-1=
		\frac{1}{2}\left(\dfrac{x_{n}}{\sqrt{a}}+\dfrac{a}{x_{n}\sqrt{a}}-2\right)=
		\frac{1}{2}\left(\dfrac{x_{n}^2+a-2x_n\sqrt{a}}{\sqrt{a}x_n}\right)=
		\frac{1}{2}\left(\dfrac{(x_n-\sqrt a)^2}{\sqrt{a}x_n}\right)
	\end{equation*}
	Ну это очевидно равно $\dfrac{\varepsilon_n^2}{2(1+\varepsilon_n)}$\\
Теперь покажем что $\varepsilon_{n+2}\leq \dfrac{1}{2}\min{(\varepsilon_{n+1}^2,\varepsilon_{n+1})}$
\begin{equation*}
	\varepsilon_{n+2}=\dfrac{\varepsilon_{n+1}^2}{2(1+\varepsilon_{n+1})}\Rightarrow \dfrac{\varepsilon_{n+1}^2}{(1+\varepsilon_{n+1})} \lessgtr \min{(\varepsilon_{n+1}^2,\varepsilon_{n+1})}
\end{equation*}
Рассмотрим три случая:
\begin{enumerate}
	\item $0< \varepsilon_{n+1} <1 \Rightarrow \varepsilon_{n+1}>\varepsilon_{n+1}^2 \Rightarrow \dfrac{\varepsilon_{n+1}^2}{(1+\varepsilon_{n+1})} \lessgtr \varepsilon_{n+1}^2 \Rightarrow \dfrac{1}{1+\varepsilon_{n+1}}\lessgtr1 \Rightarrow \dfrac{1}{1+\varepsilon_{n+1}}<1$
	\item $\varepsilon_{n+1} =1 \Rightarrow \dfrac{1}{1+1}<1$
	\item $\varepsilon_{n+1} >1 \Rightarrow \dfrac{\varepsilon_{n+1}^2}{(1+\varepsilon_{n+1})}\lessgtr \varepsilon_{n+1} \Rightarrow
		\varepsilon_{n+1}^2\lessgtr \varepsilon_{n+1}^2+1 \Rightarrow 0 <1$
\end{enumerate}
Ну и мое любимое
\begin{center}
	\textbf{ЧТД}
\end{center}
\begin{figure}[H]
	\begin{center}
		\includegraphics[scale=0.5]{1.png}
	\end{center}
\end{figure}
Видно что ошибка уменьшается квадратично, скорость сходимости зависит от начального значения.
\end{enumerate}
\section{\textbf{Метод Бакхшали}}
\begin{equation*}
	x_1=\frac{1}{2}\left(x_0+\dfrac{a}{x_0}\right) \Rightarrow x_2 =\frac{1}{4}\left(x_0+\dfrac{a}{x_0}\right)+\dfrac{a}{\left(x_0+\dfrac{a}{x_0}\right)}=\dfrac{x_0^4+6ax_0+a^2}{4x_0^2+4ax_0}
\end{equation*}
Посчитаем одну итерацию второго метода:
\begin{equation*}
	x_{n+1}=x_n+\dfrac{a-x_n^2}{2x_n}-\dfrac{(a-x_n^2)^2}{8\left(x_n+\dfrac{a-x_n^2}{2x_n}\right)x_n^2}=
	\dfrac{x_n^2+a}{2x_n}-\dfrac{(a-x_n^2)^2}{8\left(x_n+\dfrac{a-x_n^2}{2x_n}\right)x_n^2}=
	\dfrac{x_0^4+6ax_0+a^2}{4x_0^2+4ax_0}
\end{equation*}
\begin{figure}[H]
	\begin{center}
		\includegraphics[scale=0.5]{2.png}
	\end{center}
\end{figure}



\section{\textbf{Интерактивный метод с двумя переменными}}
\end{document}

\documentclass{report}

\usepackage[T2A]{fontenc}
\usepackage[russian]{babel}
\usepackage{graphicx}
\usepackage{float}
\usepackage{hyperref}
\usepackage{amsmath}
\usepackage{diffcoeff,amssymb}
\usepackage{mathtools}
\usepackage[normalem]{ulem}

\input{preamble}
\input{macros}
\input{letterfonts}

\setcounter{secnumdepth}{0}
\title{\Huge{Матан Лаба}}
\author{\huge{Павел Андреев, Григорий Горбушкин, Евгений Турчанин}}
\date{}
\begin{document}
\maketitle

\section{\textbf{Вавилонский метод}}
\begin{enumerate}
    \item Покажем, что данная последовательность сходится куда надо.
    Предположим, что она действительно сходится, тогда пусть $A$ - то, куда она сходится, АК предел, тогда:
    \begin{equation*}
    A=\frac{1}{2}\left(A+\dfrac{a}{A}\right)\Rightarrow A=\sqrt{a}
    \end{equation*}
    Покажем, что данная последовательность вообще сходится, для этого нам нужна монотонность и ограниченность:
    \begin{enumerate}
        \item Монотонность
    \begin{equation*}
        x_n-x_{n-1}=\frac{1}{2}\left(x_{n-1}+\dfrac{a}{x_{n-1}}\right)-x_{n-1}=-\frac{1}{2}x_{n-1}+\frac{1}{2}\dfrac{a}{x_{n-1}}
    \end{equation*}
    Для монотонности нужно показать больше или меньше нуля это выражение, для этого сравним два числа:
    \begin{equation*}
        \frac{1}{2}x_{n-1}\lessgtr\frac{1}{2}\dfrac{a}{x_{n-1}}\Rightarrow x_{n-1}^2\lessgtr a \Rightarrow \text{тк}\;x_{n-1}>\sqrt{a} \Rightarrow \text{Убывает}
    \end{equation*}
            Покажем в явном виде, что $x_{n-1}>\sqrt{a}$, по теореме Каши\sout{на}
            \begin{equation*}
        x_{n-1}=\frac{1}{2}\left(x_{n-2}+\dfrac{a}{x_{n-2}}\right) \geq 2\sqrt{\frac{1}{4}x_{n-2}\cdot\dfrac{a}{x_{n-2}}}=\sqrt{a}
        \end{equation*}
    \item Ограниченность
    \end{enumerate}
    Отсюда получаем и ограниченность, тк если каждый члем последовательности меньше $\sqrt{a}$, то последовательность ограниченна снизу.\sout{И вообще кто молодец? Я молодец! Правильно? Правильно!}

\item Из выше написанного следует что $\dfrac{x_n}{\sqrt{a}}-1\geq 0$\\
    Покажем что $\varepsilon_{n+1}=\dfrac{\varepsilon_n^2}{2(1+\varepsilon_n)}$:
    \begin{equation*}
        \varepsilon_{n+1}=\dfrac{x_{n+1}}{\sqrt{a}}-1=\frac{1}{2\sqrt{a}}\left(x_{n}+\dfrac{a}{x_{n}}\right)-1=
        \frac{1}{2}\left(\dfrac{x_{n}}{\sqrt{a}}+\dfrac{a}{x_{n}\sqrt{a}}-2\right)=
        \frac{1}{2}\left(\dfrac{x_{n}^2+a-2x_n\sqrt{a}}{\sqrt{a}x_n}\right)=
        \frac{1}{2}\left(\dfrac{(x_n-\sqrt a)^2}{\sqrt{a}x_n}\right)
    \end{equation*}
    Ну это очевидно равно $\dfrac{\varepsilon_n^2}{2(1+\varepsilon_n)}$\\
    
Теперь покажем что $\varepsilon_{n+2}\leq \dfrac{1}{2}\min{(\varepsilon_{n+1}^2,\varepsilon_{n+1})}$
\begin{equation*}
    \varepsilon_{n+2}=\dfrac{\varepsilon_{n+1}^2}{2(1+\varepsilon_{n+1})}\Rightarrow \dfrac{\varepsilon_{n+1}^2}{(1+\varepsilon_{n+1})} \lessgtr \min{(\varepsilon_{n+1}^2,\varepsilon_{n+1})}
\end{equation*}
Рассмотрим три случая:
\begin{enumerate}
	\item $0< \varepsilon_{n+1} <1 \Rightarrow \varepsilon_{n+1}>\varepsilon_{n+1}^2 \Rightarrow \dfrac{\varepsilon_{n+1}^2}{(1+\varepsilon_{n+1})} \lessgtr \varepsilon_{n+1}^2 \Rightarrow \dfrac{1}{1+\varepsilon_{n+1}}\lessgtr1 \Rightarrow \dfrac{1}{1+\varepsilon_{n+1}}<1$
	\item $\varepsilon_{n+1} =1 \Rightarrow \dfrac{1}{1+1}<1$
	\item $\varepsilon_{n+1} >1 \Rightarrow \dfrac{\varepsilon_{n+1}^2}{(1+\varepsilon_{n+1})}\lessgtr \varepsilon_{n+1} \Rightarrow
		\varepsilon_{n+1}^2\lessgtr \varepsilon_{n+1}^2+1 \Rightarrow 0 <1$
\end{enumerate}
Ну и мое любимое
\begin{center}
	\textbf{ЧТД}
\end{center}
\begin{figure}[H]
	\begin{center}
		\includegraphics[scale=0.5]{12.png}
	\end{center}
\end{figure}
\section{\textbf{Метод Бакхшали}}
\begin{equation*}
	x_1=\frac{1}{2}\left(x_0+\dfrac{a}{x_0}\right) \Rightarrow x_2 =\frac{1}{4}\left(x_0+\dfrac{a}{x_0}\right)+\dfrac{a}{\left(x_0+\dfrac{a}{x_0}\right)}=\dfrac{\left(\left(\dfrac{x_0^2+a}{x_0}\right)^2+4a\right)x_0}{4(x_0^2+a)}=
	\dfrac{(x_0^2+a)^2+4ax_0^2}{4x_0(x_0^2+a)}=\boxed{\dfrac{x_0^4+6x_0^2a+a^2}{4x_0^3+4ax_0}}
\end{equation*}
Посчитаем одну итерацию второго метода:
\begin{equation*}
	x_{1}^*=x_0+\dfrac{a-x_0^2}{2x_0}-\dfrac{(a-x_0^2)^2}{8\left(x_0+\dfrac{a-x_0^2}{2x_n}\right)x_0^2}=
	\dfrac{x_0^2+a}{2x_0}-\dfrac{(a-x_0^2)^2}{8\left(x_0+\dfrac{a-x_0^2}{2x_0}\right)x_0^2}=
	\dfrac{x_0^2+a}{2x_0}-\dfrac{(a-x_0^2)^2}{2x_0\left(4x_0^2+2(a-x_0^2)\right)}=
\end{equation*}
\begin{equation*}
	\dfrac{(x_0^2+a)(2x_0^2+2a)-(a-x_0^2)^2}{2x_0\left(2x_0^2+2a\right)}=\dfrac{2x_0^4+4ax_0^2+2a^2-(a^2-2ax_0^2+x_0^4)}
	{4x_0^3+4x_0a}=\boxed{\dfrac{x_0^4+6ax_0^2+a^2}{4x_0^3+4ax_0}}
\end{equation*}
\begin{figure}[H]
	\begin{center}
		\includegraphics[scale=0.5]{2.png}
	\end{center}
\end{figure}
\end{enumerate}

\section{\textbf{Интерактивный метод с двумя переменными}}
\begin{enumerate}
    \item Покажем связь между $c_{n+1}$ и $c_n$:
    \begin{equation*}
        1 + c_{n+1} \quad ?\quad (1 + c_n)\left(1 - \frac{c_n}{2}\right)^2
    \end{equation*}
    
    Раскроем правую часть:
    \begin{equation*}
        1 + \frac{c_n^2(c_n - 3)}{4}\quad ?\quad (1 + c_n)\left(1 - c_n + \frac{c_n^2}{4}\right)
    \end{equation*}
    
    \begin{equation*}
        1 + \frac{c_n^3}{4} - \frac{3c_n^2}{4} \quad ?\quad 1 - c_n + \frac{c_n^2}{4} + c_n - c_n^2 + \frac{c_n^3}{4}
    \end{equation*}
    
    \begin{equation*}
        1 + \frac{c_n^3}{4} - \frac{3c_n^2}{4} \:?\: 1 + \frac{c_n^3}{4} - \frac{3c_n^2}{4}
    \end{equation*}
    \sout{Теперь все стало очевидно}

    \item Докажем по индукции соотношение $a(1+c_n) = a_n^2$:
    \begin{enumerate}
        \item База индукции:\\
        При $n = 0$: $a(1+a-1)=a^2_0 \Rightarrow a^2=a^2$ - верно
        \item Индукционный переход:\\
        Предположим, что утверждение верно для $n = k$, докажем для $n = k+1$:
        \begin{equation*}
            a(1 + c_{k+1}) = a_{k+1}^2
        \end{equation*}
        \begin{equation*}
            a\left(1 + \frac{c_k^2(c_k - 3)}{4}\right) \quad?\quad \left(a_k - \frac{a_kc_k}{2}\right)^2
        \end{equation*}
        \begin{equation*}
            a + \frac{a c_k^3}{4} - \frac{3ac_k^2}{4} \quad?\quad a_k^2 - a_k^2c_k + \frac{a_k^2c_k^2}{4}
        \end{equation*}
        \begin{equation*}
            a + \frac{a c_k^3}{4} - \frac{3ac_k^2}{4} \quad?\quad a(1 + c_k) - a(c_k + c_k^2) + a\left(\frac{c_k^2}{4} + \frac{c_k^3}{4}\right)
        \end{equation*}
        \begin{equation*}
            a + \frac{a c_k^3}{4} - \frac{3ac_k^2}{4} \quad?\quad a + ac_k - ac_k - ac_k^2 + \frac{ac_k^2}{4} + \frac{ac_k^3}{4}
        \end{equation*}
        \begin{equation*}
            a + \frac{a c_k^3}{4} - \frac{3ac_k^2}{4} \quad?\quad a + \frac{a c_k^3}{4} - \frac{3ac_k^2}{4}
        \end{equation*}
	\sout{Теперь все стало еще очевиднее}
    \end{enumerate}

Докажем, что $c_n \rightarrow 0 \Leftrightarrow a_{n}\rightarrow \sqrt{a}$:
	\begin{equation*}
        \lim_{n\rightarrow\infty}{(a(1 + c_{n+1}))} = \lim_{n\rightarrow\infty}{(a_{n+1}^2)}
    \end{equation*}
    \begin{equation*}
        \lim_{n\rightarrow\infty}{(a)} + \lim_{n\rightarrow\infty}{(a c_{n+1})} = \lim_{n\rightarrow\infty}{(a_{n+1}^2)}
    \end{equation*}
    \begin{equation*}
        a + 0 = \lim_{n\rightarrow\infty}{(a_{n+1}^2)}
    \end{equation*}
    \begin{equation*}
        \lim_{n\rightarrow\infty}{(a_{n+1})} = \sqrt{a}
    \end{equation*}
\item
	Найдем все возможные пределы данной последовательности:
	\begin{equation*}
		A=\dfrac{A^2(A-3)}{4} \Rightarrow A_1=0; \: A_2=-1; \: A_3=4
        \end{equation*}
Покажем, что -1 и 4 не могут быть пределами последовательности при $-1<c<2$:
\begin{enumerate}
    \item Если -1 - предел, то с некого номера все члены лежать 'рядом' с -1, рассмотрим такой $-1<c_n<0$, тогда
    \begin{equation*}
        c_n \quad?\quad \dfrac{c_n^2(c_n-3)}{4}
    \end{equation*}
    \begin{equation*}
          0 \quad?\quad \dfrac{c_n(c_n+1)(c_n-4)}{4}
    \end{equation*}
    Понятно, что это число положительное, те $c_n>\dfrac{c_n^2(c_n-3)}{4}$, те последовательность возрастает, те -1 не может быть пределом
\item Если 4 - предел:
    По аналогии, если 4 - предел, то с некого номера все члены лежать 'рядом' с 4, рассмотрим такой $3<c_n<4$, тогда
        \begin{equation*}
            c_n \quad?\quad \dfrac{c_n^2(c_n-3)}{4}
        \end{equation*}
        \begin{equation*}
            0 \quad?\quad \dfrac{c_n(c_n+1)(c_n-4)}{4}
        \end{equation*}
Понятно, что это число отрицательное, те $c_n<\dfrac{c_n^2(c_n-3)}{4}$, те последовательность убывает, те 4 не может быть пределом\\
        \sout{Если говорить прям занудно} Для каждого из этих случав, если пределом является -1/4, то можно выделить бесконечную подпоследовательсноть, которая не будет идти к -1/4
    \item 
        Покажем, что 0 является пределом, по критерию Коши:
        \[
            \forall \varepsilon > 0 \; \exists n_0: \; \forall n > n_0, \, p: \; |x_{n+p} - x_n| < \varepsilon
        \]
        Понятно, что если мы рассмотрим $|x_{n+p} - x_n|$, как уравнение, то у него всегда будет корень 0, те для любого $p$ 0 является корнем, те ноль является пределом, тк если последовательность идет к нуню, то выполнятеся критерий Коши

\end{enumerate}
\end{enumerate}
\begin{figure}[H]
\begin{center}

\includegraphics[scale=0.5]{21.png}
\end{center}
\end{figure}
\end{document}

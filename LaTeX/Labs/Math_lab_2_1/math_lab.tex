\documentclass{report}

\usepackage[T2A]{fontenc}
\usepackage[russian]{babel}
\usepackage{graphicx}
\usepackage{float}
\usepackage{hyperref}
\usepackage{amsmath}
\usepackage{diffcoeff,amssymb}
\usepackage{mathtools}
\usepackage[normalem]{ulem}
\usepackage{physics}
\usepackage{wrapfig}
\usepackage{float} % для позиционирования [H]
\usepackage{siunitx} % для градусов и единиц измерения

\input{preamble}
\input{macros}
\input{letterfonts}

\setcounter{secnumdepth}{0}

\title{\Huge{Матан Лаба, вариант №18}}
\author{\huge{Григорий Горбушкин, Евгений Турчанин}}
\date{}
\begin{document}
\maketitle

\qs{}{
    Составьте интегральную сумму для функции $e^{3x}$ на отрезке $[0, 0.5]$
}
\noindent Введем равномерное разбиение отрезка $[0, 0.5]$ на $n$ частей, то есть
\begin{equation}
    x_k = \frac{k}{2n}, \quad k = 1, \ldots, n.
\end{equation}
Тогда интегральная сумма будет иметь вид
\begin{equation}
    S_n = \sum_{k=1}^{n} f(x_k) \cdot \Delta x_k = \frac{1}{2n} \sum_{k=1}^{n} e^{\frac{3k}{2n}},
\end{equation}
Перепишем сумму для правых прямоугольников, для левых прямоугольников и для средних прямоугольников:
\begin{equation}
    S_\text{правая} = \frac{1}{2n} \sum_{k=1}^{n} e^{\frac{3k}{2n}},
\end{equation}

\begin{equation}
    S_{\text{левая}} = \frac{1}{2n} \sum_{k=0}^{n-1} e^{\frac{3k}{2n}},
\end{equation}

\begin{equation}
    S_{\text{средняя}} = \frac{1}{2n} \sum_{k=1}^{n} e^{3\frac{2k-1}{4n}},
\end{equation}

\begin{equation}
    S_{\text{трапеции}} = \frac{1}{4n} \sum_{k=1}^{n} e^{3\frac{k}{2n}}+e^{3\frac{k-1}{2n}} =
    \frac{1}{4n}\left(e^{\frac{3}{2n}}+1 \right)\sum_{k=1}^{n} e^{\frac{3(k-1)}{2n}}.
\end{equation}

\qs{}{
Вычислите пределы интегральных сумм при $n \to \infty$.
}

\begin{enumerate}
    \item $S_{\text{правая}} = \displaystyle\lim_ {n \to +\infty} \dfrac{3^{\frac{3}{2n}}\cdot(e^{3/2}-1)}{2n(e^{3/2n}-1)} = \dfrac{e^{3/2}-1}{3}$,
    \item $S_{\text{левая}} = \displaystyle\lim_ {n \to +\infty} \dfrac{(e^{3/2}-1)}{2n(e^{3/2n}-1)} = \dfrac{e^{3/2}-1}{3}$,
    \item $S_{\text{средняя}} = \displaystyle\lim_ {n \to +\infty} \dfrac{e^{\frac{3}{4n}}(e^{3/2}-1)}{2n(e^{3/2n}-1)} = \dfrac{e^{3/2}-1}{3}$,
    \item $S_{\text{трапеции}} = \displaystyle\lim_ {n \to +\infty} \dfrac{(e^{3/2n}+1)(e^{3/2}-1)}{4n(e^{3/2n}-1)} = \dfrac{e^{3/2}-1}{3}$.

\end{enumerate}
\qs{}{
Докажите, что интеграл существует
}
\noindent Функция $e^{3x}$ непрерывна на отрезке $[0, 0.5]$, значит, по теореме о существовании интеграла Римана, интеграл существует.

\qs{}{
Проверьте вычисление при помощи формулы Ньютона-Лейбница
}
\[
\int _0^{0.5} e^{3x} \, dx = \left[ \frac{e^{3x}}{3} \right]_0^{0.5} = \frac{e^{3/2}-1}{3}
\]
\qs{}{
    Вывести формулу для оценки погрешности.
}
\noindent Докажем формулы для погрешности:
\begin{enumerate}
    \item Для правых прямоугольников покажем, что $|R_n| \le \displaystyle \max_{x\in[a, b]}|f'(x)|\dfrac{(b-a)^2}{2n}$.\\
        По Тейлору, для $x_k \in [x_{k}, x_{k+1}]$ найдется такое $\xi_k \in (x_{k}, x_{k+1})$, что $f(x) = f(x_{k})+f'(\xi_k)(x-x_{k})$, тогда
\[
    \int _{x_{k}}^{x_{k+1}} f(x) \, \dl x = f(x_{k})(x_{k+1}-x_{k}) + \int _{x_{k}}^{x_{k+1}} f'(\xi_k)(x-x_{k}) \, \dl x
\]
отсюда
\[
    \left|\int_{x_{k}}^{x_{k+1}}f \dl x - f(x_{k})\Delta x_k\right| \le \max_{\Delta_k} |f'(\xi_k)| \cdot \frac{(\Delta x_k)^2}{2}
\]
Если $\Delta x_k = (b-a)/n$, то
\[
    \left|\int_{a}^{b}f \dl x - \sum_{k=1}^{n} f(x_{k})\Delta x_k\right| \le \sum_{k=1}^n \max_{\Delta_k} |f'(\xi_k)| \cdot \frac{(b-a)^2}{2n^2} \le \max_{[a, b]}|f'(x)|\dfrac{(b-a)^2}{2n}
\]

\item Для средних прямоугольников, покажем что $|R_n| \le \max_{x\in [a, b]}|f''(x)|\dfrac{(b-a)^3}{24n^2}$.\\
Опять разложим в ряд Тейлора, но уже в окресности средний точки, те вокруг $\dfrac{x_k+x_{k-1}}{2}$
\[
    \int_{x_{k-1}}^{x_k} f(x) \, \dl x = f(x_{\text{ср}})(x_k-x_{k-1}) + \int_{x_{k-1}}^{x_k} f'(x_{\text{ср}})(x-x_{\text{cр}}) \dl x + \int_{x_{k-1}}^{x_k} \dfrac{f''(\xi_k)(x-x_{\text{ср}})^2}{2} \, \dl x.
\]
Попробуем обосновать разложение до второго порядка \sout{кроме фразы, что в формуле есть вторая производная}. Видно, что второй член зануляется (хотя бы из соображения симметрии), поэтому чтобы вычислить погрешность нужно раскладываться до 2-го порядка.
\[
    \left|\int_{x_{k-1}}^{x_k} f(x) \, \dl x - f(x_{\text{ср}})(x_k-x_{k-1})\right| \le \max_{\Delta_k} |f''(\xi_k)| \cdot \frac{(x_k-x_{k-1})^3}{24}.
\]
Если $\Delta x_k = (b-a)/n$, то
\[
    \left|\int_{a}^{b} f(x) \, \dl x - \sum_{k=1}^{n} f(x_{\text{ср}})\Delta x_k\right| \le \sum_{k=1}^{n} \max_{\Delta_k} |f''(\xi_k)| \cdot \frac{(x_k-x_{k-1})^3}{24} \le \max_{[a, b]}|f''(x)|\dfrac{(b-a)^3}{24n^2}.
\]
\item Для трапеций, покажем что $|R_n| \le \max_{x\in [a, b]}|f''(x)|\dfrac{(b-a)^3}{12n^2}$.\\
    Разложим в ряд Тейлора в окрестности $x_{k-1}$

\[
f(x_k) = f(x_{k-1})+f'(x_{k-1})(x_k-x_{k-1})+\dfrac{f''(\xi_k)(x_k-x_{k-1})^2}{2},
\]
теперь \sout{подгоним} подстравим это в формулу для трапеции

\[
\dfrac{x_k-x_{k-1}}{2}(f(x_k)+f(x_{k-1})) = \dfrac{x_k-x_{k-1}}{2}\left(2f(x_{k-1})+f'(x_{k-1})(x_k-x_{k-1})+\dfrac{f''(\xi_k)(x_k-x_{k-1})^2}{2})\right),
\]
распишем интеграл в его \sout{prime} форме
\[
\int _{x_{k-1}}^{x_k} f(x) \, \dl x = f(x_{k-1})\Delta x_k + \int _{x_{k-1}}^{x_k} f'(x_{k-1})(x-x_{k-1}) \, \dl x + \int _{x_{k-1}}^{x_k} \dfrac{f''(\xi_k^*)(x-x_{k-1})^2}{2} \, \dl x.
\]
Тогда их разность будет равна
\[
    \left|\int_{x_{k-1}}^{x_k} f(x) \, \dl x - \dfrac{x_k-x_{k-1}}{2}(f(x_k)+f(x_{k-1}))\right| = \left|f''(\xi_k^*)\dfrac{x_k-x_{k-1}}{6} - f''(\xi_k)\dfrac{x_k-x_{k-1}}{4}\right|
\]
тогда если $\Delta x_k = (b-a)/n$, то
\[
    \left|\int_{a}^{b} f(x) \, \dl x - \sum_{k=1}^{n} \dfrac{x_k-x_{k-1}}{2}(f(x_k)+f(x_{k-1}))\right| \le \max_{\xi \in [a, b]}|f''(\xi)|\dfrac{(b-a)^3}{12n^2}.
\]

\end{enumerate}

\qs{}{
    Найдите погрешность оценки, сравните ее с теоретической погрешностью

}
\begin{enumerate}
\item
\noindent Погрешность для правых/левых прямоугольников
\[
    |R_n| \le \max_{x\in[a, b]}|f'(x)|\dfrac{(b-a)^2}{2n} = \max_{x\in[0, 0.5]}|3e^{3x}|\dfrac{0.5^2}{2n} = \dfrac{3e^{1.5}}{8n},
\]
тогда для n=10 погрешность будет равна $0.168$, для n=100 погрешность будет равна $0.0168$
\item 
\noindent Погрешность для средних прямоугольников

\[
|R_n| \le \max_{x\in [a, b]}|f''(x)|\dfrac{(b-a)^3}{24n^2} = \max_{x\in[0, 0.5]}|9e^{3x}|\dfrac{0.5^3}{24n^2} = \dfrac{9e^{1.5}}{192n^2}
\]
тогда для n=10 погрешность будет равна $0.042$, для n=100 погрешность будет равна $0.00042$
\item 
\noindent Погрешность для трапеций
\[
|R_n| \le \max_{x\in [a, b]}|f''(x)|\dfrac{(b-a)^3}{12n^2} = \max_{x\in[0, 0.5]}|9e^{3x}|\dfrac{0.5^3}{12n^2} = \dfrac{9e^{1.5}}{96n^2}
\]
тогда для n=10 погрешность будет равна $0.063$, для n=100 погрешность будет равна $0.00063$
\begin{figure}[H]
    \begin{center}
    \includegraphics[width=0.8\textwidth]{mis_1.png}
\end{center}
\end{figure}


\dots дальше Григорий

\section{\textbf{ Маятник.}}

\subsection{Малые колебания}

\subsubsection{Исследование теоретического изменения угла от практического}

Для эксперимента была использована нить, длинной $L = 0.94 \text{ м}$. В таком случае, теоретическая частота $\omega = \sqrt{\dfrac{g}{L}} =  \sqrt{\dfrac{9.82}{0.94}} \approx 3.23 \text{ } c^{-1}$

Тогда, теоретическая зависимость угла от времени: $\theta = \theta_0\cos{(\omega t)}$

Начальные углы в каждой серии: 


\begin{table}[H]
    \centering
    \begin{tabular}{|p{2cm}|p{2cm}|p{2cm}|p{2.5cm}|p{2.5cm}|p{2.5cm}|}
        \hline 
        1 серия (малые углы) & 2 серия (малые углы) & 3 серия (малые углы) & 
        1 серия (большие углы) & 2 серия (большие углы) & 3 серия (большие углы) \\ 
        \hline 
        $\theta_0 = \ang{14.61}$ & $\theta_0 = \ang{6.22}$ & $\theta_0 = \ang{13.69}$ & 
        $\theta_0 = \ang{54.72}$ & $\theta_0 = \ang{55.00}$ & $\theta_0 = \ang{43.20}$ \\ 
        \hline
    \end{tabular}
    \caption{Начальные углы}
    \label{tab:value}
\end{table}

\begin{figure}[H]
    \begin{center}
    \includegraphics[width=1.11\textwidth]{1.png}
    \caption{1 серия}
\end{center}
\end{figure}
\begin{figure}[H]
    \begin{center}
    \includegraphics[width=1.11\textwidth]{2.png}
    \caption{2 серия}
\end{center}
\end{figure}
\begin{figure}[H]
    \begin{center}
    \includegraphics[width=1.11\textwidth]{3.png}
    \caption{3 серия}
\end{center}
\end{figure}

\subsubsection{Исследование изменения теоретического периода от практического}

Теоретически, период колебаний будет рассчитываться как $T = 2\pi\sqrt{\dfrac{l}{g}} = 1.94 \text{ с}$

Для серий эти периоды были равны (вычисляем как время 5 колебаний деленное на 5):

$T_1 \approx 2.10 \text{ с}$
$T_2 \approx 2.19 \text{ с}$
$T_3 \approx 2.01 \text{ с}$

\subsubsection{Объяснение перехода в (3)}

$\dot\theta = \pm\omega\sqrt{\theta_0^2-\theta^2}
\\
\dfrac{\dd\theta}{\dd t} = \pm\omega\sqrt{\theta_0^2-\theta^2}
\\
\dfrac{\dd\theta}{\sqrt{\theta_0^2-\theta^2}} = \pm\omega\dd t
\\
\int\limits_{\theta_0}^{\theta}\dfrac{\dd\theta}{\sqrt{\theta_0^2-\theta^2}} = \left.\arcsin{\dfrac{\theta}{\theta_0}} \right|_{\theta=\theta_0}^{\theta}=\arcsin{\dfrac{\theta}{\theta_0}} - \arcsin{\dfrac{\theta_0}{\theta_0}} = \arcsin{\dfrac{\theta}{\theta_0}} - \dfrac{\pi}{2}
\\
\int\limits_{0}^{t}\pm\omega\dd t = \pm\omega t \Rightarrow \dfrac{\theta}{\theta_0} = \sin{\left(\pm\omega t + \dfrac{\pi}{2}\right)} = \cos{(\omega t)} \text{ Ч.Т.Д.}
$
\subsubsection{Оценка погрешностей}

У нас имеется три серии, в который теоретически период должен быть одинаковый. Среднее значение практического периода колебания:

$\overline{T} = \dfrac{\sum T}{3} = 2.10 \text{ c}$

$\Delta t = \sqrt{t_{0.9;3}^2\dfrac{\sum(T_i-\overline{T})^2}{6}+\Delta t_{\text{пр}}^2} = 0.18 \text{ c}$

$\varepsilon_t = 8.6 \text{ \%}$

Тогда, значение $T$ с доверительным интервалом:

$T = 2.10\pm 0.18 \text{ c}$

Выводы: Теоретическое значение (1,94 с) не лежит в доверительном интервале полученного практически значения. Данное отличие может быть обусловлено наличием вязкого трения, т к при наличии такого трения уменьшается частота колебаний $\Rightarrow$ увеличивается период колебаний. Кроме того, не все колебания получились "идеально" малыми - двух из трех экспериментах угол чуть больше $\ang{10}$. 

Анализируя графики различия теоретического изменения угла от практически полученного можно наблюдать, что чем дольше идет наблюдение, тем больше погрешность. Если вначале это может быть связано с приближениями мат модели, то в конце большее влияние на погрешность поведения угла оказывает как раз различие в периоде.

\subsection{Нелинейные колебания}

\subsubsection{Исследование теоретического периода от практического}

Используя значения начальных углов в таблице, найдем теоретическое значение периода колебания (вычисляем инетеграл по сумме средних прямоугольников)
\\
$
T_1=2.02  \text{ с}
\\
T_2=2.02\text{ с}
\\
T_3=2.01\text{ с}
$

%Погрешность:
% \\
% $
% f'(x) = \dfrac{\cos\left(x\right)\,\sin\left(x\right)}{\left({\cos^{2}\left(x\right)-\cos^{2}\left(\theta_0\right)}\right)^{\frac{3}{2}}}$ -- на интервале $[0;1)$ монотонно растет $\Rightarrow$ $\max{\abs{f'{(x)}}} = f'(\theta_0)$
% \\
% $
% f'(0.96) = 
% $



Графики зависимости практического изменения угла от времени:

\begin{figure}[H]
    \begin{center}
    \includegraphics[width=0.8\textwidth]{4.png}
    \caption{$\theta_0 = \ang{54.72}$}
\end{center}
\end{figure}
\begin{figure}[H]
    \begin{center}
    \includegraphics[width=0.8\textwidth]{5.png}
    \caption{$\theta_0 = \ang{55.00}$}
\end{center}
\end{figure}
\begin{figure}[H]
    \begin{center}
    \includegraphics[width=0.8\textwidth]{6.png}
    \caption{$\theta_0 = \ang{43.20}$}
\end{center}
\end{figure}



Практически полученное значение периода колебаний от времени:
\\
$T_{1_{\text{прак}}} = 2.06 \text{ с}$
\\
$T_{2_{\text{прак}}} = 2.06 \text{ с}$
\\
$T_{3_{\text{прак}}} = 2.02 \text{ с}$

Таким образом, отличие теоретического периода от практического:
\\
$
\Delta_1 = 0.04\text{ с}
\\
\Delta_2 = 0.04\text{ с}
\\
\Delta_3 = 0.01\text{ с}
$

\subsubsection{Доказательство формулы (6)}


\begin{equation}
\cos\theta - \cos\theta_0 = 2\sin^2\left(\frac{\theta_0}{2}\right) - 2\sin^2\left(\frac{\theta}{2}\right) = 2k^2 - 2\sin^2\left(\frac{\theta}{2}\right)
\label{eq:trig_identity}
\end{equation}
где введено обозначение $k \equiv \sin\left(\frac{\theta_0}{2}\right)$.


Введём новую переменную $u$ через соотношение:
\begin{equation}
\sin u = \frac{\sin\left(\frac{\theta}{2}\right)}{k}
\label{eq:substitution}
\end{equation}


Продифференцируем подстановку:
\begin{equation}
\cos u \dd{u} = \frac{\cos\left(\frac{\theta}{2}\right)}{2k} \dd{\theta}
\end{equation}

\begin{equation}
\dd{\theta} = \frac{2k \cos u}{\cos\left(\frac{\theta}{2}\right)} \dd{u}
\end{equation}
Учитывая, что $\cos\left(\frac{\theta}{2}\right) = \sqrt{1 - k^2\sin^2 u}$, получаем:
\begin{equation}
\dd{\theta} = \frac{2k \cos u}{\sqrt{1 - k^2\sin^2 u}} \dd{u}
\label{eq:differential}
\end{equation}


Используя \eqref{eq:trig_identity}, выразим знаменатель:
\begin{equation}
\sqrt{\cos\theta - \cos\theta_0} = \sqrt{2}\sqrt{k^2 - \sin^2\left(\frac{\theta}{2}\right)} = \sqrt{2}k\sqrt{1 - \sin^2 u} = \sqrt{2}k\cos u
\end{equation}



\begin{equation}
\int \frac{\dd{\theta}}{\sqrt{\cos\theta - \cos\theta_0}} = \int \frac{1}{\sqrt{2}k\cos u} \cdot \frac{2k\cos u}{\sqrt{1 - k^2\sin^2 u}} \dd{u} = \sqrt{2} \int \frac{\dd{u}}{\sqrt{1 - k^2\sin^2 u}}
\end{equation}


\begin{itemize}
\item При $\theta = 0$: $\sin u = 0 \Rightarrow u = 0$
\item При $\theta = \theta_0$: $\sin u = 1 \Rightarrow u = \frac{\pi}{2}$
\end{itemize}

Итоговое преобразование
\begin{equation}
\int\limits_0^{\theta_0} \frac{\dd{\theta}}{\sqrt{\cos\theta - \cos\theta_0}} = \sqrt{2} \int\limits_0^{\pi/2} \frac{\dd{u}}{\sqrt{1 - k^2\sin^2 u}} = \sqrt{2} F\left(\frac{\pi}{2}, k\right)
\label{eq:result}
\end{equation}
где $F(\varphi,k)$ -- неполный эллиптический интеграл первого рода.


\subsubsection{Вычисление эллиптического интеграла}

\[
K(k) = \int\limits_0^{\pi/2} \frac{du}{\sqrt{1 - k^2 \sin^2 u}} 
      = \frac{\pi}{2} \sum_{n=0}^\infty \left( \frac{(2n-1)!!}{(2n)!!} k^n \right)^2,
      \quad (-1)!! = 1.
\]

\[
K(\sin{0.96}) = \int\limits_0^{\pi/2} \frac{du}{\sqrt{1 - k^2 \sin^2 u}} 
      = \frac{\pi}{2} \left(\left( \frac{(-1)!!}{(0)!!}\cdot \sin^0(0.96) \right)^2 +\left( \frac{(2-1)!!}{(2)!!}\cdot \sin(0.96) \right)^2\right) = 1.83 \text{, n  = 1}
\]
\[
K(\sin{0.96}) = \int\limits_0^{\pi/2} \frac{du}{\sqrt{1 - k^2 \sin^2 u}} = 1.93 \text{, n  = 2}
\]
\[
K(\sin{0.96}) = \int\limits_0^{\pi/2} \frac{du}{\sqrt{1 - k^2 \sin^2 u}} = 1.98 \text{, n  = 3}
\]


\[
K(\sin{0.75}) = \int\limits_0^{\pi/2} \frac{du}{\sqrt{1 - k^2 \sin^2 u}} 
      = \frac{\pi}{2} \left(\left( \frac{(-1)!!}{(0)!!}\cdot \sin^0(0.96) \right)^2 +\left( \frac{(2-1)!!}{(2)!!}\cdot \sin(0.96) \right)^2\right) = 1.75 \text{, n  = 1}
\]
\[
K(\sin{0.75}) = \int\limits_0^{\pi/2} \frac{du}{\sqrt{1 - k^2 \sin^2 u}} = 1.80 \text{, n  = 2}
\]
\[
K(\sin{0.75}) = \int\limits_0^{\pi/2} \frac{du}{\sqrt{1 - k^2 \sin^2 u}} = 1.81 \text{, n  = 3}
\]

Можно сделать вывод, что при использовании трех слагаемых сумма стремительно приближается к вычисленному численно значению интеграла (это очень круто).

\subsubsection{Итоговые погрешности}

\[
 \left( \frac{1}{\sqrt{\cos \theta - \cos \theta_0}} \right)'' = \frac{3 \sin^2 \theta}{4 (\cos \theta - \cos \theta_0)^{5/2}} + \frac{\cos \theta}{2 (\cos \theta - \cos \theta_0)^{3/2}}
\]

Очевидно, что максимум значения этой производной при $\theta\rightarrow\theta_0$. . Тогда, для:
\\
$\theta_0 = 0.75$
\\
Чтобы избавиться от особенностей, делаем замену:
\\
$t = \sqrt{0.75-\theta}$. Тогда исходный интеграл и его вторая производная принимают следующий вид:
\\
\[
\left(\frac{t }{\sqrt{\cos(0.75 - t^2) - \cos 0.75}}\right)'' = \]
\[
f''(t) = \frac{-2u \cdot (2t \sin \phi) - t \cdot (2 \sin \phi + 4t^2 \cos \phi) \cdot u + \frac{3}{2} t \cdot (2t \sin \phi)^2}{u^{5/2}}
\]

где:
\begin{itemize}
    \item \( u = \cos(0.75 - t^2) - \cos 0.75 \),
    \item \( \phi = 0.75 - t^2 \).
\end{itemize}

В этом случае, погрешность вычисленного интеграла $\abs{R_n}\approx3\cdot10^{-7}$

Аналогично делаем для $\theta\approx0.96$. В этом случае замена: $t = \sqrt{0.75-\theta}$. В этом случае, погрешность составит $\abs{R_n} \approx 9\cdot10^{-7}$
\end{document}

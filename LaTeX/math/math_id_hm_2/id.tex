\documentclass{report}

\usepackage[T2A]{fontenc}
\usepackage[russian]{babel}
\usepackage{graphicx}
\usepackage{float}
\usepackage{hyperref}
\usepackage{amsmath}
\usepackage{diffcoeff,amssymb}
\usepackage[normalem]{ulem}

\input{preamble}
\input{macros}
\input{letterfonts}

\newcommand\smallO{
  \mathchoice
    {{\scriptstyle\mathcal{O}}}
    {{\scriptstyle\mathcal{O}}}
    {{\scriptscriptstyle\mathcal{O}}}
    {\scalebox{.7}{$\scriptscriptstyle\mathcal{O}$}}
}


\title{\Huge{Матан инд}\\ Вариант №18}

\author{\huge{Евгений Турчанин}}

\date{}
\begin{document}
\maketitle
\qs{}{
Найти площадь фигуры, ограниченную кривыми:
\begin{enumerate}
    \item $\left(x^2+y^2\right)^2 = a^2x^2$
    \item $x = 2\cos t - \cos 2t, \, y = 2\sin t - \sin 2t$
\end{enumerate}
}
\sol

\begin{enumerate}
\item Перейдем в полярные координаты:
\begin{align*}
    x &= r\cos\varphi, \\
    y &= r\sin\varphi.
\end{align*}
Тогда уравнение примет вид:
\[
    r^4 = a^2r^2\cos^2\varphi \Rightarrow r = \pm a\cos\varphi, \\
\]
Площадь фигуры в полярных координатах равна:
\[
    S = \frac{1}{2}\int\limits_{0}^{2\pi}a^2\cos^2\varphi \dl \varphi = \frac{a^2}{2}\int\limits_{0}^{2\pi}\frac{1+\cos 2\varphi}{2}\dl \varphi = \frac{a^2}{4}\left[\varphi + \frac{\sin 2\varphi}{2}\right]_0^{2\pi} = \frac{a^2\pi}{2}.
\]
\item Для поиска площади фигуры воспользуемся формулой:
\[
    S = \frac{1}{2}\int\limits_{0}^{2\pi}|x(t)y'(t) - x'(t)y(t)|\dl t.
\] 
Посчитаем производные:
\begin{align*}
    x'(t) &= -2\sin t + 2\sin 2t, \\
    y'(t) &= 2\cos t - 2\cos 2t.
\end{align*}
Тогда:

\begin{align*}
    S &= \frac{1}{2}\int\limits_{0}^{2\pi}\left(2\cos t-\cos 2t\right)\left(2\cos t - 2\cos 2t\right) - \left(-2\sin t + 2\sin 2t\right)\left(2\sin t - \sin 2t\right)\dl t =\\
    &= \frac{1}{2}\int\limits_{0}^{2\pi}\left(4\cos^2 t - 4\cos t\cos 2t - 2\cos 2t\cos t + 2\cos^2 2t + 4\sin^2 t - 2\sin t\sin 2t - 4\sin 2t\sin t + 2\sin^2 2t\right)\dl t =\\
    &= \frac{1}{2}\int\limits_{0}^{2\pi}6-6\cos 2t \cos t-6\sin 2t \sin t \dl t = 3\int\limits_{0}^{2\pi}1\dl t - 3\int\limits_{0}^{2\pi}\cos 2t \cos t \dl t - 3\int\limits_{0}^{2\pi}\sin 2t \sin t \dl t =\\
    &= 6\pi
\end{align*}
\end{enumerate}
\clm{}{}{\textbf{Ответ:}
\begin{enumerate}
    \item $\frac{a^2\pi}{2}$.
    \item $6\pi$.
\end{enumerate}
}
\qs{}{
Найти длину кривой, заданной уравнением:
\begin{enumerate}
    \item $r = \dfrac{2}{\cos^4(\varphi/4)}$
    \item $2(y^2+z^2) = x, \, z\cos 2x - y\sin 2x = 0, \, 0 \le x < \pi/4$
\end{enumerate}
}
\sol
\begin{enumerate}
\item В полярных координатах длина кривой равна:
    \begin{align*}
        L &= \int\limits_{0}^{2\pi}\sqrt{\frac{4}{\cos^8(\varphi/4)} + \dfrac{4\sin^2(\varphi/4)}{\cos^{10}(\varphi/4)}}\dl \varphi =\\
        &= \int\limits_{0}^{2\pi}\dfrac{2}{\cos^5(\varphi/4)}\dl \varphi = 
    \end{align*}
\item Выразим $y$, $z$ через $x$:
\begin{align*}

\end{align*}
\end{enumerate}
\end{document}

\documentclass{report}

\usepackage[T2A]{fontenc}
\usepackage[russian]{babel}
\usepackage{graphicx}
\usepackage{float}
\usepackage{hyperref}
\usepackage{amsmath}
\usepackage{diffcoeff,amssymb}
\usepackage{mathtools}
\usepackage[normalem]{ulem}

\input{preamble}
\input{macros}
\input{letterfonts}

% Assuming preamble.tex, macros.tex, letterfonts.tex contain standard settings.
% Minimal example settings if files are missing:
% \usepackage{geometry}
% \geometry{a4paper, margin=1in}
% \newcommand{\dl}{\,\mathrm{d}} % Differential d
% \newcommand{\qs}[3]{% Fake command for question section
%   \subsection*{Задача #1}
%   #3
% }
% \newcommand{\sol}{% Fake command for solution section
%   \subsubsection*{Решение}
% }
% \newcommand{\clm}[3]{% Fake command for answer section
%   \subsubsection*{#3}
% }
% \renewcommand{\ln}{\log} % If ln should be log
% \input{preamble}
% \input{macros}
% \input{letterfonts}



\title{\Huge{Вдв}\\ Решение дз №4}
\author{\huge{Евгений Турчанин}}
\date{}
\begin{document}
\maketitle

\qs{185}{}{
$\displaystyle \int \dfrac{dx}{x^2(x^2+1)^2}$
}
\sol
\text{Метод неопределенных коэффициентов:}
\[
\dfrac{1}{x^2(x^2+1)^2} = \dfrac{A}{x^2} + \dfrac{B}{x^2+1} + \dfrac{Cx+D}{(x^2+1)^2}
\]
\[
1 = A(x^2+1)^2 + Bx^2(x^2+1) + (Cx+D)x^2
\]
\[
1 = A(x^4+2x^2+1) + B(x^4+x^2) + Cx^3+Dx^2
\]
\[
1 = (A+B)x^4 + Cx^3 + (2A+B+D)x^2 + A
\]
\text{Сравнивая коэффициенты:}
\begin{align*}
x^0:& A = 1 \\
x^3:& C = 0 \\
x^4:& A+B = 0 \implies B = -A = -1 \\
x^2:& 2A+B+D = 0 \implies 2(1) + (-1) + D = 0 \implies 1+D = 0 \implies D = -1
\end{align*}
\text{Таким образом:}
\[
\dfrac{1}{x^2(x^2+1)^2} = \dfrac{1}{x^2} - \dfrac{1}{x^2+1} - \dfrac{1}{(x^2+1)^2}
\]
\text{Интегрируем:}
\begin{align*}
\int \dfrac{dx}{x^2(x^2+1)^2} &= \int \dfrac{1}{x^2} dx - \int \dfrac{1}{x^2+1} dx - \int \dfrac{1}{(x^2+1)^2} dx \\
&= -\dfrac{1}{x} - \arctg x - \int \dfrac{1}{(x^2+1)^2} dx
\end{align*}
\text{Для последнего интеграла используем рекуррентную формулу или подстановку} $x = \tan t$:
\[
\int \dfrac{1}{(x^2+1)^2} dx = \dfrac{x}{2(x^2+1)} + \dfrac{1}{2}\int \dfrac{1}{x^2+1} dx = \dfrac{x}{2(x^2+1)} + \dfrac{1}{2}\arctg x
\]
\text{Подставляем обратно:}
\begin{align*}
\int \dfrac{dx}{x^2(x^2+1)^2} &= -\dfrac{1}{x} - \arctg x - \left( \dfrac{x}{2(x^2+1)} + \dfrac{1}{2}\arctg x \right) + C \\
&= -\dfrac{1}{x} - \dfrac{3}{2}\arctg x - \dfrac{x}{2(x^2+1)} + C
\end{align*}
\text{Ответ из зеленой рамки на изображении (с возможной ошибкой знака):}
\[
-\dfrac{1}{x} - \arctan x + \dfrac{x}{2(x^2+1)} - \dfrac{1}{2}\arctan x + C = -\dfrac{1}{x} - \dfrac{3}{2}\arctan x + \dfrac{x}{2(x^2+1)} + C
\]


\qs{191}{}{
$\displaystyle \int \dfrac{x-1}{(x^2+x+1)^2} dx$
}
\sol
\text{Метод Остроградского:}
\[
\int \dfrac{x-1}{(x^2+x+1)^2} dx = \dfrac{ax+b}{x^2+x+1} + \int \dfrac{cx+d}{x^2+x+1} dx
\]
\text{Дифференцируем обе части:}
\[
\dfrac{x-1}{(x^2+x+1)^2} = \dfrac{d}{dx}\left( \dfrac{ax+b}{x^2+x+1} \right) + \dfrac{cx+d}{x^2+x+1}
\]
\[
\dfrac{x-1}{(x^2+x+1)^2} = \dfrac{a(x^2+x+1) - (ax+b)(2x+1)}{(x^2+x+1)^2} + \dfrac{(cx+d)(x^2+x+1)}{(x^2+x+1)^2}
\]
\[
x-1 = a(x^2+x+1) - (2ax^2+ax+2bx+b) + (cx^3+cx^2+cx+dx^2+dx+d)
\]
\[
x-1 = (a-2a)x^2+(a-a-2b)x+(a-b) + cx^3+(c+d)x^2+(c+d)x+d
\]
\[
x-1 = cx^3 + (-a+c+d)x^2 + (-2b+c+d)x + (a-b+d)
\]
\text{Сравнивая коэффициенты:}
\begin{align*}
x^3:& c = 0 \\
x^2:& -a+c+d = 0 \implies -a+d = 0 \implies a=d \\
x^1:& -2b+c+d = 1 \implies -2b+d = 1 \\
x^0:& a-b+d = -1
\end{align*}
\text{Решаем систему:}
\begin{align*}
a&=d \\
-2b+d&=1 \\
a-b+d&=-1 \implies d-b+d=-1 \implies 2d-b=-1
\end{align*}
\text{Из} $-2b+d=1 \implies d = 1+2b$. \text{Подставляем в} $2d-b=-1$:
\[
2(1+2b) - b = -1 \implies 2+4b-b = -1 \implies 3b = -3 \implies b = -1
\]
\[
d = 1+2(-1) = -1
\]
\[
a = d = -1
\]
\text{Итак,} $a=-1, b=-1, c=0, d=-1$.
\begin{align*}
\int \dfrac{x-1}{(x^2+x+1)^2} dx &= \dfrac{-x-1}{x^2+x+1} + \int \dfrac{-1}{x^2+x+1} dx \\
&= -\dfrac{x+1}{x^2+x+1} - \int \dfrac{1}{(x+1/2)^2 + 3/4} dx \\
&= -\dfrac{x+1}{x^2+x+1} - \int \dfrac{1}{(x+1/2)^2 + (\sqrt{3}/2)^2} d(x+1/2) \\
&= -\dfrac{x+1}{x^2+x+1} - \dfrac{1}{\sqrt{3}/2} \arctg \left( \dfrac{x+1/2}{\sqrt{3}/2} \right) + C \\
&= -\dfrac{x+1}{x^2+x+1} - \dfrac{2}{\sqrt{3}} \arctg \left( \dfrac{2x+1}{\sqrt{3}} \right) + C
\end{align*}

\qs{196}{}{
$\displaystyle \int \dfrac{x^2-1}{x^4+x^2+1} dx$
}
\sol
\text{Делим числитель и знаменатель на} $x^2$:
\begin{align*}
\int \dfrac{1 - 1/x^2}{x^2 + 1 + 1/x^2} dx &= \int \dfrac{1 - 1/x^2}{(x^2 + 1/x^2) + 1} dx \\
&= \int \dfrac{1 - 1/x^2}{(x+1/x)^2 - 2 + 1} dx \\
&= \int \dfrac{1}{(x+1/x)^2 - 1} d(x+1/x)
\end{align*}
\text{Замена} $t = x+1/x$:
\[
\int \dfrac{dt}{t^2-1} = \dfrac{1}{2} \ln \left| \dfrac{t-1}{t+1} \right| + C
\]
\text{Подставляем обратно} $t = x+1/x$:
\[
\dfrac{1}{2} \ln \left| \dfrac{x+1/x - 1}{x+1/x + 1} \right| + C = \dfrac{1}{2} \ln \left| \dfrac{\frac{x^2-x+1}{x}}{\frac{x^2+x+1}{x}} \right| + C = \dfrac{1}{2} \ln \left| \dfrac{x^2-x+1}{x^2+x+1} \right| + C
\]
\text{Ответ из зеленой рамки на изображении (без множителя 1/2):}
\[
\ln \left| \dfrac{x+1/x - 1}{x+1/x + 1} \right| + C
\]

\qs{170}{}{
$\displaystyle \int \dfrac{dx}{(x+1)\sqrt{x^2+1}}$
}
\sol
\text{Замена} $t = \dfrac{1}{x+1}$. \text{Тогда} $x+1 = 1/t$, $x = 1/t - 1$, $dx = -1/t^2 dt$.
\begin{align*}
x^2+1 &= (1/t - 1)^2 + 1 = \left(\dfrac{1-t}{t}\right)^2 + 1 = \dfrac{1-2t+t^2}{t^2} + \dfrac{t^2}{t^2} = \dfrac{2t^2-2t+1}{t^2} \\
(x+1)\sqrt{x^2+1} &= \dfrac{1}{t} \sqrt{\dfrac{2t^2-2t+1}{t^2}} = \dfrac{1}{t} \dfrac{\sqrt{2t^2-2t+1}}{|t|} = \dfrac{\sqrt{2t^2-2t+1}}{t^2} \quad (\text{при } t>0)
\end{align*}
\text{Интеграл:}
\begin{align*}
\int \dfrac{-1/t^2 dt}{\sqrt{2t^2-2t+1}/t^2} &= - \int \dfrac{dt}{\sqrt{2t^2-2t+1}} \\
&= - \int \dfrac{dt}{\sqrt{2(t^2-t+1/2)}} \\
&= - \dfrac{1}{\sqrt{2}} \int \dfrac{dt}{\sqrt{t^2-t+1/4+1/4}} \\
&= - \dfrac{1}{\sqrt{2}} \int \dfrac{dt}{\sqrt{(t-1/2)^2 + (1/2)^2}} \\
&= - \dfrac{1}{\sqrt{2}} \ln \left| (t-1/2) + \sqrt{(t-1/2)^2 + (1/2)^2} \right| + C \\
&= - \dfrac{1}{\sqrt{2}} \ln \left| t-\frac{1}{2} + \sqrt{t^2-t+1/2} \right| + C
\end{align*}
\text{Подставляем обратно} $t = 1/(x+1)$:
\[
= - \dfrac{1}{\sqrt{2}} \ln \left| \frac{1}{x+1}-\frac{1}{2} + \sqrt{\frac{1}{(x+1)^2}-\frac{1}{x+1}+\frac{1}{2}} \right| + C
\]
\text{Ответ из зеленой рамки на изображении (с константой 3/8 вместо 1/4):}
\[
-\dfrac{1}{\sqrt{2}} \ln \left| \dfrac{1}{x+1} - \dfrac{1}{2} + \sqrt{\left(\dfrac{1}{x+1}-\dfrac{1}{2}\right)^2 + \dfrac{3}{8}} \right| + C
\]

\qs{173}{}{
$\displaystyle \int \dfrac{dx}{(x-1)\sqrt{4x^2-10x+7}}$
}
\sol
\text{Замена} $t = \dfrac{1}{x-1}$. \text{Тогда} $x-1 = 1/t$, $x = 1/t + 1$, $dx = -1/t^2 dt$.
\begin{align*}
4x^2-10x+7 &= 4(1/t+1)^2 - 10(1/t+1) + 7 \\
&= 4\left(\frac{1+t}{t}\right)^2 - 10\left(\frac{1+t}{t}\right) + 7 \\
&= \dfrac{4(1+2t+t^2)}{t^2} - \dfrac{10t(1+t)}{t^2} + \dfrac{7t^2}{t^2} \\
&= \dfrac{4+8t+4t^2 - 10t-10t^2 + 7t^2}{t^2} = \dfrac{t^2-2t+4}{t^2} \\
(x-1)\sqrt{4x^2-10x+7} &= \dfrac{1}{t} \sqrt{\dfrac{t^2-2t+4}{t^2}} = \dfrac{\sqrt{t^2-2t+4}}{t^2} \quad (\text{при } t>0)
\end{align*}
\text{Интеграл:}
\begin{align*}
\int \dfrac{-1/t^2 dt}{\sqrt{t^2-2t+4}/t^2} &= - \int \dfrac{dt}{\sqrt{t^2-2t+4}} \\
&= - \int \dfrac{dt}{\sqrt{(t-1)^2+3}} \\
&= - \ln \left| (t-1) + \sqrt{(t-1)^2+3} \right| + C
\end{align*}
\text{Подставляем обратно} $t = 1/(x-1)$:
\begin{align*}
&= - \ln \left| \left(\frac{1}{x-1}-1\right) + \sqrt{\left(\frac{1}{x-1}-1\right)^2+3} \right| + C \\
&= - \ln \left| \frac{1-(x-1)}{x-1} + \sqrt{\frac{(1-(x-1))^2 + 3(x-1)^2}{(x-1)^2}} \right| + C \\
&= - \ln \left| \frac{2-x}{x-1} + \frac{\sqrt{(2-x)^2+3(x^2-2x+1)}}{|x-1|} \right| + C \\
&= - \ln \left| \frac{2-x + \sqrt{4-4x+x^2+3x^2-6x+3}}{x-1} \right| + C \quad (\text{при } x-1>0) \\
&= - \ln \left| \frac{2-x + \sqrt{4x^2-10x+7}}{x-1} \right| + C
\end{align*}
\text{Ответ из зеленой рамки на изображении:}
\[
- \ln \left| \frac{1}{x-1}-1 + \sqrt{\left(\frac{1}{x-1}-1\right)^2+3} \right| + C
\]


% \clm{}{}{\textbf{Ответ}
% 	\begin{enumerate}
% 		\item $-\dfrac{1}{x} - \dfrac{3}{2}\arctg x + \dfrac{x}{2(x^2+1)} + C$ % (Note: Image green box has this, calculation seems to lead to - sign for last term)
%       \item $-\dfrac{x+1}{x^2+x+1} - \dfrac{2}{\sqrt{3}} \arctg \left( \dfrac{2x+1}{\sqrt{3}} \right) + C$
%       \item $\dfrac{1}{2} \ln \left| \dfrac{x^2-x+1}{x^2+x+1} \right| + C$ % (Note: Image is missing 1/2)
%       \item $-\dfrac{1}{\sqrt{2}} \ln \left| \frac{1}{x+1}-\frac{1}{2} + \sqrt{\left(\frac{1}{x+1}-\frac{1}{2}\right)^2+\frac{1}{4}} \right| + C$ % (Note: Image has 3/8 instead of 1/4)
%       \item $- \ln \left| \frac{2-x + \sqrt{4x^2-10x+7}}{x-1} \right| + C$ % (Note: Image green box has equivalent form)
% 	\end{enumerate}
% }
\end{document}

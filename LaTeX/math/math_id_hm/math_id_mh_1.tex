\documentclass{report}

\usepackage[T2A]{fontenc}
\usepackage[russian]{babel}
\usepackage{graphicx}
\usepackage{float}
\usepackage{hyperref}
\usepackage{amsmath}
\usepackage{diffcoeff,amssymb}
\usepackage{mathtools}
\usepackage[normalem]{ulem}


\input{preamble}
\input{macros}
\input{letterfonts}
\newcommand\smallO{
  \mathchoice
    {{\scriptstyle\mathcal{O}}}% \displaystyle
    {{\scriptstyle\mathcal{O}}}% \textstyle
    {{\scriptscriptstyle\mathcal{O}}}% \scriptstyle
    {\scalebox{.7}{$\scriptscriptstyle\mathcal{O}$}}%\scriptscriptstyle
  }


\title{\Huge{Матан инд}\\ Вариант №19(22)}

\author{\huge{Евгений Турчанин}}
\date{}
\begin{document}
\maketitle

\qs{}{
Доказать что:
$\dfrac{1^4}{1\cdot3}+\dfrac{2^4}{3\cdot5}+\mathellipsis+\dfrac{n^4}{(2n-1)\cdot(2n+1)}=\dfrac{n(n+1)(n^2+n+1)}{6(2n+1)}$
}
\sol
Докажем через индукцию:\\
\begin{enumerate}
	\item Покажем что для $n=1$ верно: \\ 
		\begin{equation*}
		\dfrac{1(1+1)(1^2+1+1)}{6(2+1)}=\dfrac{1}{3}
		\end{equation*}
		\begin{equation*}
		\dfrac{1^4}{1\cdot3}=\dfrac{1}{3}
		\end{equation*}
	\item Пусть верно для $n$ \\
	\item Тогда докажем, что верно для $n+1$:
		\begin{equation*}
			\underbrace{\dfrac{1^4}{1\cdot3}+\dfrac{2^4}{3\cdot5}+\mathellipsis+\dfrac{n^4}{(2n-1)\cdot(2n+1)}}_{\dfrac{n(n+1)(n^2+n+1)}{6(2n+1)}}+\dfrac{(n+1)^4}{(2n+1)\cdot(2n+3)} \Rightarrow
		\end{equation*}
		Нужно доказать, что:
		\begin{equation*}
		\dfrac{n(n+1)(n^2+n+1)}{6(2n+1)}+\dfrac{(n+1)^4}{(2n+1)\cdot(2n+3)} = 
		\dfrac{(n+1)(n+2)(n^2+3n+3)}{6(2n+3)}
		\end{equation*}
		\sout{Трудно не заметить, что так оно и есть}
		Давайте покажем, что это так:\\
		Сократим все на $n+1$ и приведем к общему знаменателю:
		\begin{equation*}
		\dfrac{n(2n+3)(n^2+n+1)}{6(2n+1)(2n+3)}+\dfrac{6(n+1)^3}{6(2n+1)\cdot(2n+3)} = 
		\dfrac{(2n+1)(n+2)(n^2+3n+3)}{6(2n+1)(2n+3)}
		\end{equation*}
		Сократим на знаменатели и раскроем скобки:
		\begin{equation*}
		2x^4+11x^3+23x^2+21x+6=2x^3+11x^2+23x+21+6 \Rightarrow
		\end{equation*}
		\begin{equation*}
		0=0
		\end{equation*}
		\begin{center}
			Ч.Т.Д.
		\end{center}

\end{enumerate}



\qs{}{
Доказать что:
\[ \sum_{k=1}^{k}\dfrac{1}{k^2} \leqslant 2 -\dfrac{1}{n}\]
}
\sol
Докажем опять \sout{двадцатьпять} через индукцию:\\
\begin{enumerate}
	\item Покажем что для $n=1$ верно: \\ 
		\begin{equation*}
			\dfrac{1}{1^2}=1
		\end{equation*}
		\begin{equation*}
			2-\dfrac{1}{1}=1
		\end{equation*}
	\item Пусть верно для $n$ \\
	\item Тогда докажем, что верно для $n+1$:\\
		Те нужно доказать что верно:
		\begin{equation*}
			\underbrace{\dfrac{1}{1^2}+\dfrac{1}{2^2}+\mathellipsis+\dfrac{1}{n^2}}_{\leqslant 2-\frac{1}{n}}+\dfrac{1}{(n+1)^2} \leqslant 2-\dfrac{1}{n+1}
		\end{equation*}
		Приведем к общему знаменателю:
		\begin{equation*}
			\dfrac{(2n-1)(n+1)^2+n}{(n+1)^2n} \leqslant \dfrac{(2n+1)(n+1)n}{(n+1)^2n}
		\end{equation*}
		Сократим на знаменатель:
		\begin{equation*}
			(2n-1)(n+1)^2+n\leqslant (2n+1)(n+1)n
		\end{equation*}
		Раскроем скобки:
		\begin{equation*}
			2n^3+4n^2+2n-n^2-2n-1+n\leqslant 2n^3+2n^2+n^2+n \Rightarrow
		\end{equation*}
		\begin{equation*}
			-1\leqslant 0
		\end{equation*}
		\begin{center}
			Ч.Т.Д.
		\end{center}

\end{enumerate}


\begin{center}
	\textbf{\Large{Момент когда меня сместили с 19 на 22 место}}
\end{center}
\hrulefill

\qs{}{
Доказать:
$(2n-1)!<n^{2n-1}, \; n \geq 2$

}
\sol
Докажем по мат. индукции:
\begin{enumerate}
	\item Покажем что для $n=2$ верно: \\
		\begin{equation*}
			(4-1)!<2^{4-1} \Rightarrow 6<8
		\end{equation*}

	\item Пусть верно для $n$ \\
	\item Докажем, что верно для $n+1$ \\
		\begin{equation*}
			(2n+1)!<(n+1)^{2n+1} \Rightarrow 2n(2n+1)n^{2n-1}<(n+1)^{2n+1} \Rightarrow 
		\end{equation*}
		\begin{equation*}
			4n^{2n+1}+2n^{2n}<n^{2n+1}+(n+1)^{2n}\cdot(2n+1)+(n+1)^{2n-1}\cdot n(2n+1)
		\end{equation*}
		Преобразуем правую часть:
		\begin{equation*}
			n^{2n+1}+(n+1)^{2n}\cdot(2n+1)+(n+1)^{2n-1}\cdot n(2n+1)>5n^{2n+1}+2n^{2n}
		\end{equation*}
		Тк
		\begin{equation*}
			4n^{2n+1}+2n^{2n}<5n^{2n+1}+2n^{2n}
		\end{equation*}
		\begin{center}
		\textbf{Ч.Т.Д.}
		\end{center}
\end{enumerate}

\qs{}{
$C_n^0-2C_n^1+3C_n^2+\ldots +(-1)^n(n+1)C_n^n$ = ?
}
\sol
\sout{Легко}Трудно не заметить что:\\

\begin{equation*}
C_n^0-2C_n^1+3C_n^2+\ldots+(-1)^n(n+1)C_n^n=(n+1)C_n^0-nC_n^1+(n-1)C_n^2+\ldots+(-1)^nC_n^n
\end{equation*}
\\
Тогда начальную сумму можно представить ввиде:
\begin{equation*}
\dfrac{n+1}{2}(C_n^0-C_n^1+(-1)^nC_n^2+\ldots+C_n^n)
\end{equation*}

\begin{equation*}
C_n^0-C_n^1+C_n^2+\ldots+(-1)^nC_n^n=(a+b)^n
\end{equation*}
\\
Такое равенство будет выполнятся при $a=1$ и $b=-1$(причем, очевидно что четность/нечетность $n$ не влияет), тогда искомая сумма будет равна:
\begin{equation*}
\dfrac{n+1}{2}0^n=0
\end{equation*}

\clm{}{}{\textbf{Ответ\sout{ственность}:}
$C_n^0-2C_n^1+3C_n^2+\ldots +(-1)^n(n+1)C_n^n=0$
}
\hfill

\qs{}{
Доказать:
\begin{enumerate}
	\item $\lim\limits_{n\to\infty} \dfrac{3n-3}{6n-4}=\dfrac{1}{2}$
	\item $\lim\limits_{n\to\infty} (3-\ln{(n+1)})=-\infty$
	\item $\lim\limits_{n\to\infty} ((-1)^n\cdot n^2-n)=\infty$
\end{enumerate}
}
\sol

\begin{enumerate}
	\itemОпределение предела:
		\begin{equation*}
			\forall \epsilon > 0 \: \exists n_0 \in \mathbb{N}: \forall n \geq n_0: |A-x_n|< \epsilon
		\end{equation*}
		Тогда чтобы доказать что $1/2$ является пределом, нужно показать что:
		\begin{equation*}
			\left|\dfrac{1}{2}-\dfrac{3n-3}{6n-4}\right|<\epsilon
		\end{equation*}
		Те что $\left|\dfrac{1}{2}-\dfrac{3n-3}{6n-4}\right|$ может быть сколь угодно малым, покажем это:
		\begin{equation*}
			\left|\dfrac{1}{2}-\dfrac{3n-3}{6n-4}\right|=\dfrac{1}{6n-4} - \text{эта дробь может быть сколь угодно малой из принципа Архимеда}
		\end{equation*}
		\begin{center}
			\textbf{Ч.Т.Д.}
		\end{center}
	\item Если пределом последовательности является $-\infty$, то не существует такого числа, которое \textit{подпирает} последовательность снизу и последовательность монотонно убывает. Запишем это в формальном виде:
		\begin{equation*}
			\forall M \enspace \exists x_{n_0} : \forall n \geq n_0: x_n< M
		\end{equation*}
		Докажем что для любого $M$ можно найти такой $x_{n_0}$, начиная с которого, все члены мешьше $M$:
		\begin{equation*}
			3-\ln(n+1)<M \Rightarrow n>e^{3-M}-1  \text{--- ну вот он, можно округлить до целых и прибавить 1}
		\end{equation*}
		Теперь покажем монотонность последовательности:
		\begin{equation*}
			x_{n+1}-x_n=3-\ln{(n+2)}-3+\ln{(n+1)}=\ln{\dfrac{n+1}{n+2}}<0 \Rightarrow \text{последовательность монотонно убывает}
		\end{equation*}
		Получаем что, последовательность монотонно убывает и не ограничена снизу
		\begin{center}
			\textbf{Ч.Т.Д.}
		\end{center}
	\item Если предел последовательност это $\infty$, то множество ее частичных пределов --- $\lbrace-\infty, \infty\rbrace$\\
	Тогда докажем, что $\pm \infty$ являются частичными пределами, и что других частичных пределов нету:
	\begin{enumerate}
		\item $n$ - четное $\Rightarrow$
			$\lim\limits_{n\to\infty} (n^2-n)=\lim\limits_{n\to\infty} n^2(1-1/n) \to \infty$

		\item $n$ - нечетное $\Rightarrow$
			$\lim\limits_{n\to\infty} (-n^2-n)=\lim\limits_{n\to\infty} -n^2(1+1/n) \to -\infty$
	\end{enumerate}
	Других частичных пределов нету, тк в подпоследовательность входит либо бесконечное число четных и нечетных, тогда предел --- $\infty$, либо конечное число четных/нечетных и бесконечное число нечетных/четных, тогда с какого-то номера в подпоследовательности не будет четных/нечетных чисел, тогда частичне пределы --- $\mp \infty$
	\begin{center}
		\textbf{Ч.Т.Д.}
	\end{center}
\end{enumerate}

\qs{}{
$
\begin{gathered}
	1)\lim_{n\to\infty}\frac{5n\sqrt[\leftroot{0}\uproot{2}4]{n^3-2n}+n\sqrt{n^2+2n+8}}{\sqrt[\leftroot{0}\uproot{2}3]{n^6-5n+4}-3n^2-5n}; 2) \lim_{n\to\infty}\frac{3^{2n-1}+2^{n+5}+4n^5}{n\sqrt{3n+5}-9^{n+3}}; \\
3) \lim_{n\to\infty}\left(\sqrt{4n^4+3n^2+5}-\sqrt{4n^4-6n^2-7}\right); 4) \lim_{n\to\infty}\sqrt[n]{3+\frac{4}{2n-7}}; \\
5) \lim_{n\to\infty}\left(\frac{7n+15}{9n+8}\right)^{3n}; 6) \lim_{n\to\infty}\left(\frac{6n-5}{6n+7}\right)^{3n}; 7) \lim_{n\to\infty}\frac{n^{2}-(n+1)!}{n!\cdot(n+4)+5^{n}}; \\
8) \lim_{n\to\infty}\sqrt[n]{3n^{2}+\frac{4}{2n-7}}; 9) \lim_{n\to\infty}\frac{\sqrt[\leftroot{0}\uproot{2}n]{2}-1}{\sqrt[\leftroot{0}\uproot{2}n]{8}-1}; 10) \lim_{n\to\infty}\frac{3+6+9+...+3n}{n^{2}+4}. 
\end{gathered}
$
}
\begin{enumerate}
	\item $
		\lim\limits_{n\to\infty}\dfrac{5n\sqrt[\leftroot{0}\uproot{2}4]{n^3-2n}+n\sqrt{n^2+2n+8}}{\sqrt[\leftroot{0}\uproot{2}3]{n^6-5n+4}-3n^2-5n}=\lim\limits_{n\to\infty}\dfrac{5\sqrt[\leftroot{0}\uproot{2}4]{n^3-2n}+n\sqrt{1+2/n+8/n^2}}{n\sqrt[\leftroot{0}\uproot{2}3]{1-5/n^5+4/n^6}-n(3+5/n)}=\\
		\\
		\lim\limits_{n\to\infty}\dfrac{n\sqrt{1+2/n+8/n^2}\left(\dfrac{5\sqrt[\leftroot{0}\uproot{2}4]{n^3-2n}}{n\sqrt{1+2/n+8/n^2}}+1\right)}{n\sqrt[\leftroot{0}\uproot{2}3]{1-5/n^5+4/n^6}-n(3+5/n)}=
		\lim\limits_{n\to\infty}\dfrac{\sqrt{1+2/n+8/n^2}\left(\dfrac{5\sqrt[\leftroot{0}\uproot{2}4]{n^3-2n}}{n\sqrt{1+2/n+8/n^2}}+1\right)}{\sqrt[\leftroot{0}\uproot{2}3]{1-5/n^5+4/n^6}-(3+5/n)}=\\
		\lim\limits_{n\to\infty}\dfrac{\sqrt{1+2/n+8/n^2}\left(\dfrac{5\sqrt[\leftroot{0}\uproot{2}4]{n^3-2n}}{n\sqrt{1+2/n+8/n^2}}+1\right)}{(3+5/n)\left(\dfrac{\sqrt[\leftroot{0}\uproot{2}3]{1-5/n^5+4/n^6}}{3+5/n}-1\right)}=\dfrac{1\cdot1}{3\cdot\left(\dfrac{1}{3}-1\right)}=-\dfrac{1}{2}
	$
	\item $
		\lim\limits_{n\to\infty}\dfrac{3^{2n-1}+2^{n+5}+4n^5}{n\sqrt{3n+5}-9^{n+3}}=\lim\limits_{n\to\infty}\dfrac{9^{n}/3+2^{n}\cdot32+4n^5}{n\sqrt{3n+5}-9^{n}\cdot729}=
		\lim\limits_{n\to\infty}\dfrac{9^{n}/3\left(1+\dfrac{2^{n}\cdot32}{9^n/3}+\dfrac{4n^5}{9^n/3}\right)}{9^{n}\cdot729\left(\dfrac{n\sqrt{3n+5}}{9^n\cdot729}-1\right)}=\dfrac{1/3\cdot1}{729\cdot-1}=-\dfrac{1}{2187}
	$
\item $
	\lim\limits_{n\to\infty}\sqrt{4n^4+3n^2+5}-\sqrt{4n^4-6n^2-7}=\lim\limits_{n\to\infty}\dfrac{9n^2+12}{\sqrt{4n^4+3n^2+5}+\sqrt{4n^4-6n^2-7}}=\\
	\lim\limits_{n\to\infty}\dfrac{9n^2+12}{2n^2\sqrt{1+3/4n^2+5/4n^4}\left(1+\dfrac{\sqrt{4n^4-6n^2-7}}{\sqrt{4n^4+3n^2+5}}\right)}=\lim\limits_{n\to\infty}\dfrac{9n^2+12}{2n^2\sqrt{1+3/4n^2+5/4n^4}\left(1+\sqrt{1+\dfrac{-9n^2-9}{4n^4+3n^2+5}}\right)}=\dfrac{9}{2\cdot(1+1)}=\dfrac{9}{4}
$
\item $
	\lim\limits_{n\to\infty}\sqrt[n]{3+\dfrac{4}{2n-7}}=1 \quad \text{тк корень из чего-то положительного(что растет медленее корня) $\to$ 1}
$
\item $
\lim\limits_{n\to\infty}\left(\dfrac{7n+15}{9n+8}\right)^{3n}=\lim\limits_{n\to\infty}\left(\dfrac{7(1+15/n)}{9(1+8/n)}\right)^{3n}=0
$
\item $
	\lim\limits_{n\to\infty}\left(\dfrac{6n-5}{6n+7}\right)^{3n}=\lim\limits_{n\to\infty}\left(1+\dfrac{-12}{6n+7}\right)^{3n}=\lim\limits_{n\to\infty}e^{\frac{-12}{6n+7}\cdot 3n}=e^{-6}
$
\item $
	\lim\limits_{n\to\infty}\dfrac{n^{2}-(n+1)!}{n!\cdot(n+4)+5^{n}}=\lim\limits_{n\to\infty}\dfrac{(n+1)!\left(\dfrac{n^2}{(n+1)!}-1\right)}{n!(n+4)\left(1+\dfrac{5^n}{n!(n+4)}\right)}=
\lim\limits_{n\to\infty}\dfrac{-(n+1)}{n+4}=-1
$
\item $
\lim\limits_{n\to\infty}\sqrt[n]{3n^{2}+\dfrac{4}{2n-7}}=1 \quad \text{по аналогии с 4}
$
\item $
	\lim\limits_{n\to\infty}\dfrac{\sqrt[\leftroot{0}\uproot{2}n]{2}-1}{\sqrt[\leftroot{0}\uproot{2}n]{8}-1} \quad \text{используя $\sqrt[\leftroot{0}\uproot{2}n]{a}\approx1+\dfrac{\ln a}{n}$ при $n \to \infty$} \Rightarrow
	\lim\limits_{n\to\infty}\dfrac{\sqrt[\leftroot{0}\uproot{2}n]{2}-1}{\sqrt[\leftroot{0}\uproot{2}n]{8}-1}=\dfrac{\ln 2}{\ln 8}
$
\item $
\lim\limits_{n\to\infty}\dfrac{3+6+9+...+3n}{n^{2}+4}=\lim\limits_{n\to\infty}\dfrac{3n(n+1)}{n^{2}+4}=\lim\limits_{n\to\infty}\dfrac{3n^2(1+3/n)}{n^{2}(1+4/n^2)}=3
$
\end{enumerate}
\clm{}{}{\textbf{Ответ:}
\begin{enumerate}
	\item $-\dfrac{1}{2}$
	\item $-\dfrac{1}{2187}$
	\item $\dfrac{9}{4}$
	\item 1
	\item 0
	\item $e^{-6}$
	\item -1
	\item 1
	\item $\dfrac{\ln 2}{\ln 8}$
	\item 3
\end{enumerate}
}

\qs{}{
	Найти $\sup x_n,\: \inf x_n,\: \varliminf\limits_{n\to\infty}x_n,\: \varlimsup\limits_{n\to\infty}x_n$\\
	$x_n=\dfrac{2n+1}{n}\arccos{(-1)^n}$
}
\sol
\parindent ДлДля начала рассмотрим последовательность при четных $n$:\\
\[
	x_n=\dfrac{2n+1}{n}\arccos{1}=0 \quad - \text{\sout{офигенная}хорошая подпоследовательность}
\]
Теперь рассмотрим при нечетных $n$:\\
\[
	x_n=\dfrac{2n+1}{n}\pi=2+\dfrac{1}{n}\pi\quad-  \text{понятно, что это убыв. подпоследовательность}
\]
Понятно, что других подпоследовательностей нету, тк в последовательности может быть 3 случая:
\begin{enumerate}
	\item Бесконечное количество четных и нечетных номеров, тогда такая подпоследовательность не сход, тк нельзя найти номер после которого $\varepsilon$ можно взять любым, тк при четных можно найти 0, а при нечетных найти >2
	\item При конечном количестве четных и бесконечном количестве нечетных, можно найти номер с которого будут только нечетные, те подпоследовательность сход к $2$
	\item По аналогии с 2, но подпоследовательность сход к 0
\end{enumerate}
Теперь мы готовы сказать, что $\sup x_n = 3\pi$, $\inf x_n = 0$, $\varliminf\limits_{n\to\infty}x_n = 0$, $\varlimsup\limits_{n\to\infty}x_n = 2$ --- из определения $\varlimsup\limits_{n\to\infty}x_n = \lim\limits_{n\to\infty}(\sup_{m\geq n}x_m)$
\clm{}{}{\textbf{Ответ:}
\begin{enumerate}
	\item $\sup x_n = 3\pi$
	\item $\inf x_n = 0$
	\item $\varliminf\limits_{n\to\infty}x_n = 0$
	\item $\varlimsup\limits_{n\to\infty}x_n = 2$
\end{enumerate}
}
\qs{}{
	Определить сходится ли последовательность:\\
	\begin{center}
	$x_n=x_{n-1}+\dfrac{1}{2^n}$
	\end{center}
}
\sol
По критерию Коши: Если последовательность фундаментальна т.е.
\begin{equation}
	\forall \varepsilon > 0 \: \exists n_0:\: \forall n \geq n_0 \:\exists p \quad |x_{n+p}-x_n|<\varepsilon
\end{equation}
То последовательность сходится, для нашей последовательности возьмем p=1:
\begin{equation}
	x_n+\dfrac{1}{2^{n+1}}-x_n=\dfrac{1}{2^{n+1}}
\end{equation}
Понятно, что $\dfrac{1}{2^{n+1}}$ может быть сколь угодно малым $\Rightarrow$ последовательность сходится

\qs{}{
Используя признак Вейерштрасса, доказать, что данная последовательность сходится, и найти ее предел\\
\begin{center}
$x_n=\dfrac{5}{2}\cdot\dfrac{9}{7}\cdot\dfrac{13}{12}...\dfrac{4n+1}{5n-3}$
\end{center}
}
\sol
Нашу последовательность можно переписать ввиде:
\begin{equation*}
	x_{n}=x_{n-1}\dfrac{4n+1}{5n-3}, \; n_1=\dfrac{5}{2}
\end{equation*}
Тогда покажем, что она огр. снизу и монотонно убывает:
\begin{equation*}
	\dfrac{x_n}{x_{n-1}}=\dfrac{4n+1}{5n-3}  \end{equation*}
Очевидно, что при $n>3$:
\begin{equation*}
\dfrac{4n+1}{5n-3}<1
\end{equation*}
Наша последовательность убывает, и очевидно она огр. снизу 0, тк каждый член получается путем, умножения положительных чисел
Тогда найдем предел:
\begin{equation*}
	A=A\cdot \dfrac{4n+1}{5n-3} \Rightarrow A=0
\end{equation*}
Те наша последовательность имеет предел 0
\clm{}{}{\textbf{Ответ:}
\\
\\
$x_n=\dfrac{5}{2}\cdot\dfrac{9}{7}\cdot\dfrac{13}{12}...\dfrac{4n+1}{5n-3}\to0$
}
\qs{}{
Найти сумму ряда:
\[
	\sum_{n=1}^{\infty}\dfrac{12}{4n^2-8n-5}
\]
}
\sol
Через метод неопределенных коэффициентов найдем разложение на две дроби:
\[
	\sum_{n=1}^{\infty}\dfrac{12}{(2n+1)(2n-5)}=\sum_{n=1}^{\infty}\dfrac{2}{2n-5}-\sum_{n=1}^{\infty}\dfrac{2}{2n+1}
\]
Трудно не заметить, что \[\sum_{n=1}^{\infty}\dfrac{2}{2n-5}=\sum_{n=2}^{\infty}\dfrac{2}{2n+1}=\sum_{n=1}^{\infty}\dfrac{2}{2n+1}-\dfrac{2}{3} \]
Тогда \[ \sum_{n=1}^{\infty}\dfrac{12}{4n^2-8n-5}=-\dfrac{2}{3}\]

\clm{}{}{\textbf{Ответ:}
\\
$\displaystyle \sum_{n=1}^{\infty}\dfrac{12}{4n^2-8n-5}=-\dfrac{2}{3}$

}


\qs{}{
Доказать по определению предела функции в точке (по Коши):
\[
	\lim\limits_{x\to1} \dfrac{5x^2-4x-1}{x-1}=6
\]
}
\sol
Из определения предела по Коши:
\begin{equation*}
\forall \varepsilon > 0 \enspace \exists \delta > 0: \forall x \in \mathbb{R}: \left| x-1\right|<\delta \Rightarrow \left|\dfrac{5x^2-4x-1}{x-1}-6\right|<\varepsilon
\end{equation*}
Раскроем модуль, тогда:
\begin{equation*}
\left|\dfrac{5x^2-4x-1}{x-1}-6\right|<\varepsilon \Rightarrow \left|\dfrac{5x^2-10x+5}{x-1}\right| \Rightarrow \left|x-1 \right|<\dfrac{\varepsilon}{5}
\end{equation*}
Понятно, что если мы возьмем $\delta=\dfrac{\varepsilon}{5}$, то все ок

\qs{}{
Доказать, что данный предел не существует:
\[
\lim\limits_{x\to +\infty} \tg x
\]
}
\sol
Чтобы показать, что предела не существует, нужно найти две последовательности, которые сходятся в разные точки:
\begin{equation*}
	x_{n_1}=\dfrac{\pi}{4}+\pi k, \enspace k \in \mathbb{Z} \quad \enspace
	x_{n_2}=\pi q, \enspace q \in \mathbb{Z}
\end{equation*}
Понятно, что $x_{n_1}$ идет к 1, а $x_{n_2}$ идет к 0, те \sout{беспредел} предела не существует

\qs{}{
Вычислить пределы:
\begin{enumerate}
\item $\lim\limits_{x\to1}\dfrac{\left(2x^{2}-x-1\right)^{2}}{x^{3}+2x^{2}-x-2}$
\item $\lim\limits_{x\to-8}\dfrac{10-x-6\sqrt{1-x}}{2+\sqrt[\leftroot{0}\uproot{2}3]{x}}$
\item $\lim\limits_{x\to\pi/4}\dfrac{\ln\mathrm{tg}x}{\cos2x}$
\item $\lim\limits_{x\to2}\dfrac{\ln\left(x-\sqrt[\leftroot{0}\uproot{2}3]{2x-3}\right)}{\sin\left(\pi x/2\right)-\sin\left[\left(x-1\right)\pi\right]}$
\item $\lim\limits_{x\to0}\left(1-\sin^2\dfrac{x}{2}\right)^{1/\ln\left(1+\mathrm{tg}^23x\right)}$
\item $\lim\limits_{x\to1\pm0}\left(\dfrac{2x-1}{x+1}\right)^{1/\left(\sqrt[\leftroot{0}\uproot{2}3]{x}-1\right)}$
\item $\lim\limits_{x\to\pi/4}\dfrac{\sqrt[\leftroot{0}\uproot{2}3]{\mathrm{tg}x}+(4x-\pi)\cos\dfrac{x}{4x-\pi}}{\lg\left(2+\mathrm{tg}x\right)}$
\item $\lim\limits_{x\to+0}\dfrac{\ln\left(1-\ln x\right)}{\ln\left(1-\lg x\right)}$
\end{enumerate}
}

\sol
\begin{enumerate}
\item
$
\lim\limits_{x\to1}\dfrac{\left(2x^{2}-x-1\right)^{2}}{x^{3}+2x^{2}-x-2}=\lim\limits_{x\to1}\dfrac{(2x+1)^2(x-1)}{(x+2)(x+1)}=0
$
\item
По правилу Лопиталя:\\
$
\lim\limits_{x\to-8}\dfrac{10-x-6\sqrt{1-x}}{2+\sqrt[\leftroot{0}\uproot{2}3]{x}}=\lim\limits_{x\to-8}\dfrac{-1+3/\sqrt{1-x}}{\dfrac{1}{3\sqrt[\leftroot{0}\uproot{2}3]{x^2}}}=0
$
\item
По правилу Лопиталя:\\
$
\lim\limits_{x\to\pi/4}\dfrac{\ln\mathrm{tg}x}{\cos2x}=\lim\limits_{x\to\pi/4}\dfrac{\dfrac{1}{\tg x}\cdot\dfrac{1}{\cos2x}}{-\sin 2x \cdot 2}=-1
$
\item
По правилу Лопиталя:\\
$
\lim\limits_{x\to2}\dfrac{\ln\left(x-\sqrt[\leftroot{0}\uproot{2}3]{2x-3}\right)}{\sin\left(\pi x/2\right)-\sin\left[\left(x-1\right)\pi\right]}=\lim\limits_{x\to2}\dfrac{\dfrac{1}{x-\sqrt[\leftroot{0}\uproot{2}3]{2x-3}}\cdot\left(1-\dfrac{1}{3}\cdot\dfrac{1}{2x-3}\cdot 2 \right)}{\cos\left({\dfrac{\pi x}{2}}\right)\pi/2-\cos((x-1)\pi)\pi}=\dfrac{2}{3\pi}
$
\item
$
\lim\limits_{x\to0}\left(1-\sin^2\dfrac{x}{2}\right)^{1/\ln\left(1+\mathrm{tg}^23x\right)}=\lim\limits_{x\to0}\exp\left[\dfrac{-\sin^2\dfrac{x}{2}}{\ln\left(1+\tg^2 3x\right)}\right]=
\lim\limits_{x\to0}\exp\left[\dfrac{-\sin(x/2)}{\dfrac{1}{1+\tg^2 3x}\cdot2\tg 3x\cdot\dfrac{1}{\cos^2 3x}\cdot 3}\right]=
$
\quad \\
$
=\lim\limits_{x\to0} \exp\left[\dfrac{-\sin x/2}{6\tg 3x}\right]=e^{-\dfrac{1}{36}}
$
\item
$
\lim\limits_{x\to1\pm0}\left(\dfrac{2x-1}{x+1}\right)^{1/\left(\sqrt[\leftroot{0}\uproot{2}3]{x}-1\right)}=\lim\limits_{x\to1\pm0} e^{\dfrac{\ln\left(\dfrac{2x-1}{x+1}\right)}{\sqrt[\leftroot{0}\uproot{2}3]{x}-1}}\Rightarrow
\lim\limits_{x\to1+0}\left(\dfrac{2x-1}{x+1}\right)^{1/\left(\sqrt[\leftroot{0}\uproot{2}3]{x}-1\right)}=0,\ \lim\limits_{x\to1-0}\left(\dfrac{2x-1}{x+1}\right)^{1/\left(\sqrt[\leftroot{0}\uproot{2}3]{x}-1\right)}=+\infty
$
\item
$
\lim\limits_{x\to\pi/4}\dfrac{\sqrt[\leftroot{0}\uproot{2}3]{\mathrm{tg}x}+(4x-\pi)\cos\dfrac{x}{4x-\pi}}{\lg\left(2+\mathrm{tg}x\right)},
$
\\
тк $\cos$ огр. можно на него \sout{положить} забить $\Rightarrow$
$
\lim\limits_{x\to\pi/4}\dfrac{\sqrt[\leftroot{0}\uproot{2}3]{\mathrm{tg}x}+(4x-\pi)\cos\dfrac{x}{4x-\pi}}{\lg\left(2+\mathrm{tg}x\right)}=
\lim\limits_{x\to\pi/4}\dfrac{\sqrt[\leftroot{0}\uproot{2}3]{\tg x}}{\lg(2+\tg x)}+\lim\limits_{x\to\pi/4}\dfrac{0}{\lg(2+\tg x)}=\dfrac{1}{\lg 3}
$
\item
$
\lim\limits_{x\to+0}\dfrac{\ln\left(1-\ln x\right)}{\ln\left(1-\lg x\right)}=\lim\limits_{x\to+0}\dfrac{\ln(\ln(e-x))}{\ln(\lg (10-x))}=1
$
\end{enumerate}

\end{document}

\documentclass{report}

\usepackage[T2A]{fontenc}
\usepackage[russian]{babel}
\usepackage{graphicx}
\usepackage{float}
\usepackage{hyperref}
\usepackage{amsmath}
\usepackage{diffcoeff,amssymb}
\usepackage{mathtools}
\usepackage[normalem]{ulem}


\input{preamble}
\input{macros}
\input{letterfonts}



\title{\Huge{Матан инд}\\ Вариант №19(22)}

\author{\huge{Евгений Турчанин}}
\date{}
\begin{document}
\maketitle

\qs{}{
Доказать что:
$\dfrac{1^4}{1\cdot3}+\dfrac{2^4}{3\cdot5}+\mathellipsis+\dfrac{n^4}{(2n-1)\cdot(2n+1)}=\dfrac{n(n+1)(n^2+n+1)}{6(2n+1)}$
}
\sol
Докажем через индукцию:\\
\begin{enumerate}
	\item Покажем что для $n=1$ верно: \\ 
		\begin{equation*}
		\dfrac{1(1+1)(1^2+1+1)}{6(2+1)}=\dfrac{1}{3}
		\end{equation*}
		\begin{equation*}
		\dfrac{1^4}{1\cdot3}=\dfrac{1}{3}
		\end{equation*}
	\item Пусть верно для $n$ \\
	\item Тогда докажем, что верно для $n+1$:
		\begin{equation*}
			\underbrace{\dfrac{1^4}{1\cdot3}+\dfrac{2^4}{3\cdot5}+\mathellipsis+\dfrac{n^4}{(2n-1)\cdot(2n+1)}}_{\dfrac{n(n+1)(n^2+n+1)}{6(2n+1)}}+\dfrac{(n+1)^4}{(2n+1)\cdot(2n+3)} \Rightarrow
		\end{equation*}
		Нужно доказать, что:
		\begin{equation*}
		\dfrac{n(n+1)(n^2+n+1)}{6(2n+1)}+\dfrac{(n+1)^4}{(2n+1)\cdot(2n+3)} = 
		\dfrac{(n+1)(n+2)(n^2+3n+3)}{6(2n+3)}
		\end{equation*}
		\sout{Трудно не заметить, что так оно и есть}
		Давайте покажем, что это так:\\
		Сократим все на $n+1$ и приведем к общему знаменателю:
		\begin{equation*}
		\dfrac{n(2n+3)(n^2+n+1)}{6(2n+1)(2n+3)}+\dfrac{6(n+1)^3}{6(2n+1)\cdot(2n+3)} = 
		\dfrac{(2n+1)(n+2)(n^2+3n+3)}{6(2n+1)(2n+3)}
		\end{equation*}
		Сократим на знаменатели и раскроем скобки:
		\begin{equation*}
		2x^4+11x^3+23x^2+21x+6=2x^3+11x^2+23x+21+6 \Rightarrow
		\end{equation*}
		\begin{equation*}
		0=0
		\end{equation*}
		\begin{center}
			Ч.Т.Д.
		\end{center}

\end{enumerate}



\qs{}{
Доказать что:
\[ \sum_{k=1}^{k}\dfrac{1}{k^2} \leqslant 2 -\dfrac{1}{n}\]
}
\sol
Докажем опять \sout{двадцатьпять} через индукцию:\\
\begin{enumerate}
	\item Покажем что для $n=1$ верно: \\ 
		\begin{equation*}
			\dfrac{1}{1^2}=1
		\end{equation*}
		\begin{equation*}
			2-\dfrac{1}{1}=1
		\end{equation*}
	\item Пусть верно для $n$ \\
	\item Тогда докажем, что верно для $n+1$:\\
		Те нужно доказать что верно:
		\begin{equation*}
			\underbrace{\dfrac{1}{1^2}+\dfrac{1}{2^2}+\mathellipsis+\dfrac{1}{n^2}}_{\leqslant 2-\frac{1}{n}}+\dfrac{1}{(n+1)^2} \leqslant 2-\dfrac{1}{n+1}
		\end{equation*}
		Приведем к общему знаменателю:
		\begin{equation*}
			\dfrac{(2n-1)(n+1)^2+n}{(n+1)^2n} \leqslant \dfrac{(2n+1)(n+1)n}{(n+1)^2n}
		\end{equation*}
		Сократим на знаменатель:
		\begin{equation*}
			(2n-1)(n+1)^2+n\leqslant (2n+1)(n+1)n
		\end{equation*}
		Раскроем скобки:
		\begin{equation*}
			2n^3+4n^2+2n-n^2-2n-1+n\leqslant 2n^3+2n^2+n^2+n \Rightarrow
		\end{equation*}
		\begin{equation*}
			-1\leqslant 0
		\end{equation*}
		\begin{center}
			Ч.Т.Д.
		\end{center}

\end{enumerate}


\begin{center}
	\textbf{\Large{Момент когда меня сместили с 19 на 22 место}}
\end{center}
\hrulefill

\qs{}{
Доказать:
$(2n-1)!<n^{2n-1}, \; n \geq 2$

}
\sol
Докажем по мат. индукции:
\begin{enumerate}
	\item Покажем что для $n=2$ верно: \\
		\begin{equation*}
			(4-1)!<2^{4-1} \Rightarrow 6<8
		\end{equation*}

	\item Пусть верно для $n$ \\
	\item Докажем, что верно для $n+1$ \\
		\begin{equation*}
			(2n+1)!<(n+1)^{2n+1} \Rightarrow 2n(2n+1)n^{2n-1}<(n+1)^{2n+1} \Rightarrow 
		\end{equation*}
		\begin{equation*}
			4n^{2n+1}+2n^{2n}<n^{2n+1}+(n+1)^{2n}\cdot(2n+1)+(n+1)^{2n-1}\cdot n(2n+1)
		\end{equation*}
		Преобразуем правую часть:
		\begin{equation*}
			n^{2n+1}+(n+1)^{2n}\cdot(2n+1)+(n+1)^{2n-1}\cdot n(2n+1)>5n^{2n+1}+2n^{2n}
		\end{equation*}
		Тк
		\begin{equation*}
			4n^{2n+1}+2n^{2n}<5n^{2n+1}+2n^{2n}
		\end{equation*}
		\begin{center}
		\textbf{Ч.Т.Д.}
		\end{center}
\end{enumerate}

\qs{}{
$C_n^0-2C_n^1+3C_n^2+\ldots +(-1)^n(n+1)C_n^n$ = ?
}
\sol
\sout{Легко}Трудно не заметить что:\\

\begin{equation*}
C_n^0-2C_n^1+3C_n^2+\ldots+(-1)^n(n+1)C_n^n=(n+1)C_n^0-nC_n^1+(n-1)C_n^2+\ldots+(-1)^nC_n^n
\end{equation*}
\\
Тогда начальную сумму можно представить ввиде:
\begin{equation*}
\dfrac{n+1}{2}(C_n^0-C_n^1+(-1)^nC_n^2+\ldots+C_n^n)
\end{equation*}

\begin{equation*}
C_n^0-C_n^1+C_n^2+\ldots+(-1)^nC_n^n=(a+b)^n
\end{equation*}
\\
Такое равенство будет выполнятся при $a=1$ и $b=-1$(причем, очевидно что четность/нечетность $n$ не влияет), тогда искомая сумма будет равна:
\begin{equation*}
\dfrac{n+1}{2}0^n=0
\end{equation*}

\clm{}{}{\textbf{Ответ\sout{ственность}:}
$C_n^0-2C_n^1+3C_n^2+\ldots +(-1)^n(n+1)C_n^n=0$
}
\hfill

\qs{}{
Доказать:
\begin{enumerate}
	\item $\lim\limits_{n\to\infty} \dfrac{3n-3}{6n-4}=\dfrac{1}{2}$
	\item $\lim\limits_{n\to\infty} (3-\ln{(n+1)})=-\infty$
	\item $\lim\limits_{n\to\infty} ((-1)^n\cdot n^2-n)=\infty$
\end{enumerate}
}
\sol

\begin{enumerate}
	\itemОпределение предела:
		\begin{equation*}
			\forall \epsilon > 0 \: \exists n_0 \in \mathbb{N}: \forall n \geq n_0: |A-x_n|< \epsilon
		\end{equation*}
		Тогда чтобы доказать что $1/2$ является пределом, нужно показать что:
		\begin{equation*}
			\left|\dfrac{1}{2}-\dfrac{3n-3}{6n-4}\right|<\epsilon
		\end{equation*}
		Те что $\left|\dfrac{1}{2}-\dfrac{3n-3}{6n-4}\right|$ может быть сколь угодно малым, покажем это:
		\begin{equation*}
			\left|\dfrac{1}{2}-\dfrac{3n-3}{6n-4}\right|=\dfrac{1}{6n-4} - \text{эта дробь может быть сколь угодно малой из принципа Архимеда}
		\end{equation*}
		\begin{center}
			\textbf{Ч.Т.Д.}
		\end{center}
	\item Если пределом последовательности является $-\infty$, то не существует такого числа, которое \textit{подпирает} последовательность снизу и последовательность монотонно убывает. Запишем это в формальном виде:
		\begin{equation*}
			\forall M \enspace \exists x_{n_0} : \forall n \geq n_0: x_n< M
		\end{equation*}
		Докажем что для любого $M$ можно найти такой $x_{n_0}$, начиная с которого, все члены мешьше $M$:
		\begin{equation*}
			3-\ln(n+1)<M \Rightarrow n>e^{3-M}-1  \text{--- ну вот он, можно округлить до целых и прибавить 1}
		\end{equation*}
		Теперь покажем монотонность последовательности:
		\begin{equation*}
			x_{n+1}-x_n=3-\ln{(n+2)}-3+\ln{(n+1)}=\ln{\dfrac{n+1}{n+2}}<0 \Rightarrow \text{последовательность монотонно убывает}
		\end{equation*}
		Получаем что, последовательность монотонно убывает и не ограничена снизу
		\begin{center}
			\textbf{Ч.Т.Д.}
		\end{center}
	\item Если предел последовательност это $\infty$, то множество ее частичных пределов --- $\lbrace-\infty, \infty\rbrace$\\
	Тогда докажем, что $\pm \infty$ являются частичными пределами, и что других частичных пределов нету:
	\begin{enumerate}
		\item $n$ - четное $\Rightarrow$
			$\lim\limits_{n\to\infty} (n^2-n)=\lim\limits_{n\to\infty} n^2(1-1/n) \to \infty$

		\item $n$ - нечетное $\Rightarrow$
			$\lim\limits_{n\to\infty} (-n^2-n)=\lim\limits_{n\to\infty} -n^2(1+1/n) \to -\infty$
	\end{enumerate}
	Других частичных пределов нету, тк в подпоследовательность входит либо бесконечное число четных и нечетных, тогда предел --- $\infty$, либо конечное число четных/нечетных и бесконечное число нечетных/четных, тогда с какого-то номера в подпоследовательности не будет четных/нечетных чисел, тогда частичне пределы --- $\mp \infty$
	\begin{center}
		\textbf{Ч.Т.Д.}
	\end{center}
\end{enumerate}
\qs{}{
	$\begin{gathered}
		1)\lim_{n\to\infty}\frac{5n\sqrt[\leftroot{0}\uproot{2}4]{n^3-2n}+n\sqrt{n^2+2n+8}}{\sqrt[\leftroot{0}\uproot{2}3]{n^6-5n+4}-3n^2-5n}; 2) \lim_{n\to\infty}\frac{3^{2n-1}+2^{n+5}+4n^5}{n\sqrt{3n+5}-9^{n+3}}; \\
3) \lim_{n\to\infty}\left(\sqrt{4n^4+3n^2+5}-\sqrt{4n^4-6n^2-7}\right); 4) \lim_{n\to\infty}\sqrt[n]{3+\frac{4}{2n-7}}; \\
5) \lim_{n\to\infty}\left(\frac{7n+15}{9n+8}\right)^{3n}; 6) \lim_{n\to\infty}\left(\frac{6n-5}{6n+7}\right)^{3n}; 7) \lim_{n\to\infty}\frac{n^{2}-(n+1)!}{n!\cdot(n+4)+5^{n}}; \\
8) \lim_{n\to\infty}\sqrt[n]{3n^{2}+\frac{4}{2n-7}}; 9) \lim_{n\to\infty}\frac{\sqrt[\leftroot{0}\uproot{2}n]{2}-1}{\sqrt[\leftroot{0}\uproot{2}n]{8}-1}; 10) \lim_{n\to\infty}\frac{3+6+9+...+3n}{n^{2}+4}. 
\end{gathered}
$
}
\sol
\begin{enumerate}
	\item $\lim\limits_{n\to\infty} \dfrac{5n\sqrt[\leftroot{0}\uproot{2}4]{n^3-2n}+n\sqrt{n^2+2n+8}}{\sqrt[\leftroot{0}\uproot{2}3]{n^6-5n+4}-3n^2-5n}=$
\end{enumerate}
\end{document}



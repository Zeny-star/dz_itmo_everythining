\documentclass{report}

\usepackage[T2A]{fontenc}
\usepackage[russian]{babel}
\usepackage{graphicx}
\usepackage{float}
\usepackage{hyperref}
\usepackage{amsmath}
\usepackage{diffcoeff,amssymb}
\usepackage{mathtools}

\input{preamble}
\input{macros}
\input{letterfonts}



\title{\Huge{Матан}\\ Решение дз №2}
\author{\huge{Евгений Турчанин}}
\date{}

\begin{document}
\maketitle

\qs{}{
Доказать:
\begin{itemize}
  \item $(A \cup B) \cap\left(A^{c} \cup B^{c}\right)=A \cup B$
  \item $A \Delta(B \Delta C)=(A \Delta B) \Delta C$
  \item $A \cap(B \Delta C)=(A \cap B) \Delta(A \cap C)$\\
\end{itemize}
}

\sol \begin{itemize}
  \item $(A \cup B) \cap\left(A^{c} \cup B^{c}\right)=A \cup B$
	Перепишем эти выражения по определению:
	\begin{center}
	\begin{equation}
		(x \in A \vee x \in B) \cap (x \notin A \vee x \notin B)\Leftrightarrow
	\end{equation}
	\end{center}
  \item $A \Delta(B \Delta C)=(A \Delta B) \Delta C$
	\begin{center}
	\begin{equation}
	A \Delta(B \Delta C) \Leftrightarrow A \Delta (x \in B \wedge x \notin C) \vee (x \notin B \wedge x \in C) \Leftrightarrow
	\end{equation}
	\begin{equation}
		(x \in A \wedge x \in B \wedge x \notin C) \vee (x \in A \wedge x \notin B \wedge x \in C) \vee (x \notin A \wedge x \in B \wedge x \in C)
	\end{equation}
	\begin{equation}
	(A \Delta B) \Delta C \Leftrightarrow (x \in A \wedge x \notin B) \vee (x \notin A \wedge x \in B) \Delta C \Leftrightarrow
	\end{equation}
	\begin{equation}
		(x \in A \wedge x \in B \wedge x \notin C) \vee (x \in A \wedge x \notin B \wedge x \in C) \vee (x \notin A \wedge x \in B \wedge x \in C)
	\end{equation}
	Ч.Т.Д.
	\end{center}
  \item $A \cap(B \Delta C)=(A \cap B) \Delta(A \cap C)$
	\begin{center}
	\begin{equation}
	A \cap(B \Delta C)\Leftrightarrow A \cap ((x \in B \wedge x \notin C) \vee (x \notin B \wedge x \in C))\Leftrightarrow
	\end{equation}
	\begin{equation}
	(x \in A \wedge x \in B \wedge x \notin C) \vee (x \in A \wedge x \notin B \wedge x \in C)
	\end{equation}
	\begin{equation}
	(A \cap B) \Delta(A \cap C) \Leftrightarrow (x \in A \wedge x \in B)\Delta (x \in A \wedge x \in C) \Leftrightarrow
	\end{equation}
	\begin{equation}
	(x \in A \wedge x \in B \wedge x \notin C) \vee (x \in A \wedge x \notin B \wedge x \in C)
	\end{equation}
	Ч.Т.Д.

	\end{center}
\end{itemize}


\qs{}{
Найти ОФФ
\begin{itemize}
  \item $y=\log _{3+x}\left(x^{2}-1\right)$
  \item $y=\lg (\pi-2 \arctan x)$\\
\end{itemize}
}
\sol
\begin{center}
\begin{itemize}
\item $y=\log_{3+x}(x^{2}-1)$

\begin{equation}
\begin{cases}
3+x >0 \\
3+x \neq 1\\
x^{2}-1 >0
\end{cases}
\end{equation}
Решая систему, получаем:
$x \in (-3, -2) \cup (-2, -1) \cup (1, \infty)$


\item $y=\lg (\pi-2 \arctan x)$
\begin{equation}
	\pi-2 \arctan x >0 \text(- выполняется всегда) \Rightarrow x \in \mathbb{R}
\end{equation}

\clm{}{}{\bf{Ответ:} \\ $x \in (-3, -2) \cup (-2, -1) \cup (1, \infty)\\ x \in \mathbb{R}$
}
\end{itemize}
\end{center}



\qs{}{
Найти область значений
\begin{itemize}
  \item $y=\sqrt{8-2 x-x^{2}}$
  \item $y=\sin ^{4} x+\cos ^{4} x$\\
\end{itemize}
}

\sol 
\begin{center}
\begin{itemize}
\item $y=\sqrt{8-2 x-x^{2}}$\\
\begin{equation}
	-x^2-2x+8 - \text{парабола ветвями вниз, найдем ее max}: x_{max}=\frac{-b}{2a}=-1 \Rightarrow y_{max}=3 \Rightarrow y \in [0, 3]
\end{equation}
\item $y=\sin ^{4} x+\cos ^{4} x$\\
\begin{equation}
\sin ^{4} x+\cos ^{4} x =(1-\cos^2 x)^2 +\cos^4 x = 1-2\cos^2 x+\cos^4 x 
\end{equation}
Пусть $t=\cos^2 x$, тогда
\begin{equation}
	y=2t^2-2t+1 \Rightarrow y_{min}=\frac{1}{2}; \quad y_{max}=1
\end{equation}
\end{itemize}
\end{center}
\clm{}{}{\bf{Ответ:} \\ $y \in [0, 3]\\
y \in [\frac{1}{2}, 1]$ }


\qs{}{
Доказать, что функции $f$ и $g$ взаимно обратные:
$$
f=x^{2}+1, x \leq 0, g=-\sqrt{x-1}, x \geq 1
$$
}


\sol Рассмотрим обзасть значений $f$:

\begin{equation}
	f(x) \in [1, \infty) \text{ такая же область значений у x во второй функции}
\end{equation}
Теперь рассмотрим область значений $g$:

\begin{equation}
	g(x) \in (-\infty, 0] \text{ такая же область значений у x в первой функции}
\end{equation}

К тому же f и g монотонный $\Rightarrow$ f и g взаимно обратные, тк выполняются инъекция и сюръекция.\\
\begin{center}
	Ч.Т.Д.
\end{center}


\end{document}

